\section{An ordinary formatting procedure}
A code formatter that takes an unformatted source code file as its input and
produces a formatted source code file as its output typically operates in three main steps:
\begin{enumerate}
  \item \textit{lexing} (also known as \textit{tokenizing} or \textit{lexical analysis})
  \item \textit{parsing} (also known as \textit{syntactic analysis})
  \item \textit{printing} (also known as \textit{unparsing}\autocite{prettyprinting})
\end{enumerate}

\subsection{Lexing}
\textit{Lexing} transforms a source code string to an array of tokens,
such as keywords, identifiers, literals and operators.
For the input string
\begin{minted}{c}
  if(a==b){c=a+2*b;}
\end{minted}
in a C-like language, it might return the tokens:
\begin{minted}{text}
  if, (, a, ==, b, ), {, c, =, a, +, 2, *, b, ;, }
\end{minted}

This step is not exclusive to formatters; lexing is a common operation that is also performed by
compilers and other static analysis tools that operate on input source code.
However, lexers often differ slightly depending on the purpose of a tool.
For example, a compiler might completely ignore comments in the source code and
not create any tokens from them, because the final machine code instructions it emits
would not contain comments anyway.

Some lexers also generate virtual tokens that are not directly present as characters in the source code.
The Python language reference describes a lexing procedure that generates
\code{INDENT} and \code{DEDENT} tokens in order to preserve information about line indentation.
\autocite[Chapter: 2.1.8. (Lexical Analysis --- Line Structure --- Indentation)]{pythonLangRef}
The Go programming language specification describes a lexing procedure
that generates semicolon tokens automatically according to a few simple rules,
\autocite[Chapter: Lexical Elements --- Semicolons]{goProgLangSpec}
while the ECMAScript language specification describes a more complicated
automatic semicolon insertion mechanism for the parsing step.
\autocite[Chapter: 11.9 (Lexical Grammar --- Automatic Semicolon Insertion)]{ecmascriptSpec}

\subsection{Parsing}
\textit{Parsing} transforms an array of tokens to a \textit{parse tree} (also known as
\textit{concrete syntax tree}) or, more commonly, to an \textit{abstract syntax tree} (\textit{AST}).
The former still contains all of the information about the tokens in the original source code,
while the latter is on a higher level and only describes
the structure of the program represented by the source code.

This difference can also have an impact on formatting, depending on the individual tree format.
A formatter that uses a parse tree might preserve the brackets in the expression
\mintinline{c}{1 + (2 * 3)}, while a formatter that uses an abstract syntax tree might
not be able to do so, because all it sees is a multiplication expression inside of an addition expression,
not whether the original source was \mintinline{c}{1 + 2 * 3} or \mintinline{c}{1 + (2 * 3)}
or \mintinline{c}{((1) + ((2) * ((3))))}.
Therefore, the formatter would always output the first variation,
regardless of any parentheses occurring in the input source code;
or it would always output the second variation in case it is configured to generate code
that displays the precedence of multiplication over addition more clearly.

\begin{figure}
  \centering \footnotesize \setminted{fontsize=\scriptsize}
  \begin{forest}
    [{Program}
    [ {IfStmt}, edge label={node[midway,left]{\code{body[0]}}}
    [  {BinaryExpr (op: \code{==})}, edge label={node[midway,left]{\code{test}}}
    [   {Identifier (name: \code{a})}, edge label={node[midway,left]{\code{left}}}]
    [   {Identifier (name: \code{b})}, edge label={node[midway,right]{\code{right}}}]]
    [  {BlockStmt}, edge label={node[midway,right]{\code{consequent}}}
    [   {ExprStmt}, edge label={node[midway,left]{\code{body[0]}}}
    [    {AssignmentExpr (op: \code{=})}, edge label={node[midway,left]{\code{expression}}}
    [     {Identifier (name: \code{c})}, edge label={node[midway,left]{\code{left}}}]
    [     {BinaryExpr (op: \code{+})}, edge label={node[midway,right]{\code{right}}}
    [      {Identifier (name: \code{a})}, edge label={node[midway,left]{\code{left}}}]
    [      {BinaryExpr (op: \code{*})}, edge label={node[midway,right]{\code{right}}}
    [       {Literal (value: \code{2})}, edge label={node[midway,left]{\code{left}}}]
    [       {Identifier (name: \code{b})}, edge label={node[midway,right]{\code{right}}}]]]]]]]]
  \end{forest}
  \caption{An \textit{abstract syntax tree} for a short if statement,
    based on the ESTree Spec \autocite{estreeSpec} for ECMAScript ASTs.
    `Statement' is abbreviated `Stmt', `Expression' is abbreviated `Expr'
  and `operator' is abbreviated `op' in order to save space.}\label{fig:ifStmtAst}
\end{figure}

For the input tokens
\begin{minted}{text}
  if, (, a, ==, b, ), {, c, =, a, +, 2, *, b, ;, }
\end{minted}
in a C-like language, a parser might return the abstract syntax tree shown in figure~\ref{fig:ifStmtAst}.

\subsection{Printing}
\textit{Printing} transforms an abstract syntax tree (or parse tree)
to a string of source code that matches the desired format.

For the input AST shown in figure~\ref{fig:ifStmtAst} in a C-like language,
a printer might return the code string:

\begin{minted}{c}
  if (a == b) {
    c = a + 2 * b;
  }
\end{minted}

Above example shows the first three of the four main concerns of the printing step in a code formatter,
which we will discuss in the following.

\paragraph{Spacing} is necessary because we want to see gaps for instance
between the \code{a} and \code{==} tokens as well as after the \code{)} token.
We do, however, not want to see gaps for instance
between the \code{(} and \code{a} tokens as well as between the \code{b} and \code{;} tokens.
Adding spaces where none existed in the original source code
or removing spaces where they existed in the original source code
might also change the semantics or even validity of the code.
For example, removing a space between a keyword and an identifier
would combine those tokens into a single, longer identifier.
For a less obvious example that shows how dangerous this can be, consider the following: In JavaScript,
\begin{minted}{javascript}
  console.log(1 .toString())
\end{minted}
is a valid program and prints the result of
calling the method \code{toString} on the number \code{1}, whereas
\begin{minted}{javascript}
  console.log(1.toString())
\end{minted}
is invalid, because \code{1} and \code{.} are no longer separate tokens;
instead, they now form a single number literal token (\code{1.})
that is followed immediately by an identifier (\code{toString}).

\paragraph{Indentation} helps us understand the control flow,
among other nestable structures, in the source code more easily.
We want to increase the indentation level at the start of a block statement and decrease it at the end.
In the aforementioned example, the assignment inside of the consequent block of the if statement
has its indentation level increased by one relative to the lines immediately above and below
that are not entirely contained inside the block statement.

\paragraph{Hard line breaks} force the insertion of a line break between two tokens.
We want this to happen in the obvious positions after an opening brace or semicolon and before a closing brace.
Depending on our desired format we also want to use hard line breaks in some less obvious situtations.
For example, it might be useful to apply them before switch cases in order to reformat
\begin{minted}{c}
  switch(a) {
    case 1: case 2:
      a = 3;
  }
\end{minted}
to
\begin{minted}{c}
  switch(a) {
    case 1:
    case 2:
      a = 3;
  }
\end{minted}

\bigbreak{}
The fourth aspect that was missing in the given example is necessary
because our output devices can only display characters in a row up to their finite width and
because we can only comfortably read lines up to a finite length.
\paragraph{Soft line breaks} are used in places where the formatter is allowed to break
if it helps avoid exceeding the given maximum allowed line length.
Most lines of code will contain more than one soft line break and usually more soft line breaks
than we actually need to bring the line length down below the limit.
Therefore, an algorithm is required to select those soft line breaks
that split up the line optimally by our standards.
Given the input
\begin{minted}{c}
  someVariable = someFunction(someParameter, someOtherParameter);
\end{minted}
and line size 50, do we want to print
\begin{minted}{c}
  someVariable =
    someFunction(someParameter, someOtherParameter);
\end{minted}
, breaking at the highest possible node in the AST, or
\begin{minted}{c}
  someVariable = someFunction(
    someParameter, someOtherParameter);
\end{minted}
, distributing the line lengths as evenly as possible, or
\begin{minted}{c}
  someVariable = someFunction(someParameter,
    someOtherParameter);
\end{minted}
, breaking as late as possible, or perhaps
\begin{minted}{c}
  someVariable = someFunction(
    someParameter,
    someOtherParameter,
  );
\end{minted}
, grouping the parameter list like we would group statements in a block,
which can be especially handy if the language allows us to place the trailing comma in the function call?
This concern is arguably the most complex aspect and has been subject to a substantial amount of research in the past.
\autocite{designPrettyPrintingLib}\autocite{prettierPrinter}

\bigbreak{}
Different formatters also preserve the original shape of the source code to a varying degree.
For example, it might be desirable to print empty lines
exactly in those places where the original source code contained them,
\autocite[Section: Empty lines]{prettierRationale}
because these empty lines usually function as separators between logically coherent sections of the code,
which cannot possibly be guessed from the tree structure by the printer.
