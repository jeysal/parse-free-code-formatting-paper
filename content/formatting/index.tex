\chapter{Code formatting}
\textit{Code formatting}, also known as \textit{code beautification} or \textit{prettyprinting},
has long been a staple in software development tooling.
Researchers have noticed its necessity and started working on solutions decades ago,
\autocite{syntaxDirectedPrettyprinting}
and the widespread availability of such coding assistance has increased ever since.

Some languages --- such as Java --- are commonly written in an integrated development environment (IDE).
The emergence of those full-fledged tooling packages has helped render formatting tools ubiquitous.
Other languages --- such as JavaScript --- are more commonly written using a rather lightweight editor,
often because the language is less verbose or simply for historical reasons.
But even some of those editors --- such as the web development-focused Microsoft Visual Studio Code ---
now include formatting tools in their core distributions together with
Git version control integration, syntax highlighting and other tooling that is deemed essential,
disposing the need to install an additional plugin that provides code formatting support.

\section{An ordinary formatting procedure}
A code formatter that takes an unformatted source code file as its input and
produces a formatted source code file as its output typically operates in three main steps:
\begin{enumerate}
  \item \textit{lexing} (also known as \textit{tokenizing} or \textit{lexical analysis})
  \item \textit{parsing} (also known as \textit{syntactic analysis})
  \item \textit{printing} (also known as \textit{unparsing}\autocite{prettyprinting})
\end{enumerate}

\subsection{Lexing}
\textit{Lexing} transforms a source code string to an array of tokens,
such as keywords, identifiers, literals and operators.
For the input string
\begin{minted}{c}
  if(a==b){c=a+2*b;}
\end{minted}
in a C-like language, it might return the tokens:
\begin{minted}{text}
  if, (, a, ==, b, ), {, c, =, a, +, 2, *, b, ;, }
\end{minted}

This step is not exclusive to formatters; lexing is a common operation that is also performed by
compilers and other static analysis tools that operate on input source code.
However, lexers often differ slightly depending on the purpose of a tool.
For example, a compiler might completely ignore comments in the source code and
not create any tokens from them, because the final machine code instructions it emits
would not contain comments anyway.

Some lexers also generate virtual tokens that are not directly present as characters in the source code.
The Python language reference describes a lexing procedure that generates
\code{INDENT} and \code{DEDENT} tokens in order to preserve information about line indentation.
\autocite[Chapter: 2.1.8. (Lexical Analysis --- Line Structure --- Indentation)]{pythonLangRef}
The Go programming language specification describes a lexing procedure
that generates semicolon tokens automatically according to a few simple rules,
\autocite[Chapter: Lexical Elements --- Semicolons]{goProgLangSpec}
while the ECMAScript language specification describes a more complicated
automatic semicolon insertion mechanism for the parsing step.
\autocite[Chapter: 11.9 (Lexical Grammar --- Automatic Semicolon Insertion)]{ecmascriptSpec}

\subsection{Parsing}
\textit{Parsing} transforms an array of tokens to a \textit{parse tree} (also known as
\textit{concrete syntax tree}) or, more commonly, to an \textit{abstract syntax tree} (\textit{AST}).
The former still contains all of the information about the tokens in the original source code,
while the latter is on a higher level and only describes
the structure of the program represented by the source code.

This difference can also have an impact on formatting, depending on the individual tree format.
A formatter that uses a parse tree might preserve the brackets in the expression
\mintinline{c}{1 + (2 * 3)}, while a formatter that uses an abstract syntax tree might
not be able to do so, because all it sees is a multiplication expression inside of an addition expression,
not whether the original source was \mintinline{c}{1 + 2 * 3} or \mintinline{c}{1 + (2 * 3)}
or \mintinline{c}{((1) + ((2) * ((3))))}.
Therefore, the formatter would always output the first variation,
regardless of any parentheses occurring in the input source code;
or it would always output the second variation in case it is configured to generate code
that displays the precedence of multiplication over addition more clearly.

\begin{figure}
  \centering \footnotesize \setminted{fontsize=\scriptsize}
  \begin{forest}
    [{Program}
    [ {IfStmt}, edge label={node[midway,left]{\code{body[0]}}}
    [  {BinaryExpr (op: \code{==})}, edge label={node[midway,left]{\code{test}}}
    [   {Identifier (name: \code{a})}, edge label={node[midway,left]{\code{left}}}]
    [   {Identifier (name: \code{b})}, edge label={node[midway,right]{\code{right}}}]]
    [  {BlockStmt}, edge label={node[midway,right]{\code{consequent}}}
    [   {ExprStmt}, edge label={node[midway,left]{\code{body[0]}}}
    [    {AssignmentExpr (op: \code{=})}, edge label={node[midway,left]{\code{expression}}}
    [     {Identifier (name: \code{c})}, edge label={node[midway,left]{\code{left}}}]
    [     {BinaryExpr (op: \code{+})}, edge label={node[midway,right]{\code{right}}}
    [      {Identifier (name: \code{a})}, edge label={node[midway,left]{\code{left}}}]
    [      {BinaryExpr (op: \code{*})}, edge label={node[midway,right]{\code{right}}}
    [       {Literal (value: \code{2})}, edge label={node[midway,left]{\code{left}}}]
    [       {Identifier (name: \code{b})}, edge label={node[midway,right]{\code{right}}}]]]]]]]]
  \end{forest}
  \caption{An \textit{abstract syntax tree} for a short if statement,
    based on the ESTree Spec \autocite{estreeSpec} for ECMAScript ASTs.
    `Statement' is abbreviated `Stmt', `Expression' is abbreviated `Expr'
  and `operator' is abbreviated `op' in order to save space.}\label{fig:ifStmtAst}
\end{figure}

For the input tokens
\begin{minted}{text}
  if, (, a, ==, b, ), {, c, =, a, +, 2, *, b, ;, }
\end{minted}
in a C-like language, a parser might return the abstract syntax tree shown in figure~\ref{fig:ifStmtAst}.

\subsection{Printing}
\textit{Printing} transforms an abstract syntax tree (or parse tree)
to a string of source code that matches the desired format.

For the input AST shown in figure~\ref{fig:ifStmtAst} in a C-like language,
a printer might return the code string:

\begin{minted}{c}
  if (a == b) {
    c = a + 2 * b;
  }
\end{minted}

Above example shows the first three of the four main concerns of the printing step in a code formatter,
which we will discuss in the following.

\paragraph{Spacing} is necessary because we want to see gaps for instance
between the \code{a} and \code{==} tokens as well as after the \code{)} token.
We do, however, not want to see gaps for instance
between the \code{(} and \code{a} tokens as well as between the \code{b} and \code{;} tokens.
Adding spaces where none existed in the original source code
or removing spaces where they existed in the original source code
might also change the semantics or even validity of the code.
For example, removing a space between a keyword and an identifier
would combine those tokens into a single, longer identifier.
For a less obvious example that shows how dangerous this can be, consider the following: In JavaScript,
\begin{minted}{javascript}
  console.log(1 .toString())
\end{minted}
is a valid program and prints the result of
calling the method \code{toString} on the number \code{1}, whereas
\begin{minted}{javascript}
  console.log(1.toString())
\end{minted}
is invalid, because \code{1} and \code{.} are no longer separate tokens;
instead, they now form a single number literal token (\code{1.})
that is followed immediately by an identifier (\code{toString}).

\paragraph{Indentation} helps us understand the control flow,
among other nestable structures, in the source code more easily.
We want to increase the indentation level at the start of a block statement and decrease it at the end.
In the aforementioned example, the assignment inside of the consequent block of the if statement
has its indentation level increased by one relative to the lines immediately above and below
that are not entirely contained inside the block statement.

\paragraph{Hard line breaks} force the insertion of a line break between two tokens.
We want this to happen in the obvious positions after an opening brace or semicolon and before a closing brace.
Depending on our desired format we also want to use hard line breaks in some less obvious situtations.
For example, it might be useful to apply them before switch cases in order to reformat
\begin{minted}{c}
  switch(a) {
    case 1: case 2:
      a = 3;
  }
\end{minted}
to
\begin{minted}{c}
  switch(a) {
    case 1:
    case 2:
      a = 3;
  }
\end{minted}

\bigbreak{}
The fourth aspect that was missing in the given example is necessary
because our output devices can only display characters in a row up to their finite width and
because we can only comfortably read lines up to a finite length.
\paragraph{Soft line breaks} are used in places where the formatter is allowed to break
if it helps avoid exceeding the given maximum allowed line length.
Most lines of code will contain more than one soft line break and usually more soft line breaks
than we actually need to bring the line length down below the limit.
Therefore, an algorithm is required to select those soft line breaks
that split up the line optimally by our standards.
Given the input
\begin{minted}{c}
  someVariable = someFunction(someParameter, someOtherParameter);
\end{minted}
and line size 50, do we want to print
\begin{minted}{c}
  someVariable =
    someFunction(someParameter, someOtherParameter);
\end{minted}
, breaking at the highest possible node in the AST, or
\begin{minted}{c}
  someVariable = someFunction(
    someParameter, someOtherParameter);
\end{minted}
, distributing the line lengths as evenly as possible, or
\begin{minted}{c}
  someVariable = someFunction(someParameter,
    someOtherParameter);
\end{minted}
, breaking as late as possible, or perhaps
\begin{minted}{c}
  someVariable = someFunction(
    someParameter,
    someOtherParameter,
  );
\end{minted}
, grouping the parameter list like we would group statements in a block,
which can be especially handy if the language allows us to place the trailing comma in the function call?
This concern is arguably the most complex aspect and has been subject to a substantial amount of research in the past.
\autocite{designPrettyPrintingLib}\autocite{prettierPrinter}

\bigbreak{}
Different formatters also preserve the original shape of the source code to a varying degree.
For example, it might be desirable to print empty lines
exactly in those places where the original source code contained them,
\autocite[Section: Empty lines]{prettierRationale}
because these empty lines usually function as separators between logically coherent sections of the code,
which cannot possibly be guessed from the tree structure by the printer.


\section{Eliminating the parsing step}
Many of the works on prettyprinting that have been published ---
such as the well-known one by Derek C. Oppen \autocite{prettyprinting} ---
propose algorithms that place line breaks into a stream of tokens.
Such a stream could be generated by a lexer;
a parser would only be required to obtain a tree structure according to
the syntactic grammar of the language.
But for any conventional formatting algorithm,
by the time it comes to actually applying the algorithm to a specific language,
parsing the tokens becomes necessary.
This is because it is sometimes hard and often impossible to
recognize coherent blocks in the token stream,
deduct the meaning of each token without knowledge about its greater surroundings or
retrieve any other information that could be used for
determining suitable spacing or line break positions when
going left to right over code written in any practical language.

An example of this problem arises with the ambigious meaning of the
\code{-} (minus) symbol \autocite[Chapter: Introduction]{prettyGoodFormattingPipeline},
which many languages use both as a
unary operator that computes the additive inverse of its operand and as a
binary operator that subtracts its right-hand operand from its left-hand operand.
Consider the following use of the `minus' symbol with its immediate neighbor tokens:
\begin{minted}{text}
  ), -, x
\end{minted}
One might intuitively proclaim that this minus symbol has to be a binary operator,
subtracting the value stored as identifier \code{x} from
whatever subexpression the parenthesis on the left closes.
But the syntactic grammars of most languages are not that simple;
the whole statement could be
\begin{minted}{c}
  if (a == b) -x;
\end{minted}
, or the language might allow parenthesized type casts and the statement looks like this:
\begin{minted}{c}
  int y = (int) -x;
\end{minted}

There are good reasons to parse the source code in order to
get better insight into its structure before attempting to prettyprint it,
but what if we still try to create a formatter that actually omits that parsing step,
printing out nicely formatted source code just based on the tokens recognized by the lexer?

\subsection{Potential benefits}
There is an obvious advantage such a formatter would have:

\paragraph{Efficiency}
In a quick test run performed by the author, the popular JavaScript parser `acorn'
took around three to five times as long to parse a large bundle of JavaScript code
than to just tokenize the same code.
Parsing can be a major bottleneck in the formatting pipeline,
so its elimination, especially when combined with a printing algorithm focused on efficiency,
could speed up the whole process of formatting significantly.

We can also expect the memory usage footprint of the formatter
to be lower in comparison to an ordinary formatter,
because we do not need to hold a large tree structure in the memory.

\paragraph{Streaming}
A non-parsing formatter could potentially be capable of writing the first line of output
right after reading the first few tokens of input and continuing to write quite consistently
as it progresses through the input code, because conceivable implementations would never
need to look at more than a section between two hard line breaking tokens at once.
Implementing such a capability would be way more trivial than it is for
formatters that parse the code first.

\paragraph{Adaptability}
A formatter that only has to understand the lexical grammar of a language,
but not its syntactic grammar, can be more easily adapted to format other languages,
even if their syntax is vastly different.

\subsection{Drawbacks}
\paragraph{Consistency}
On the other hand, it is clear that we will not be able to achieve the same degree of consistency
for some of the harder aspects of formatting
when the abstract syntax of expressions is the same,
but the concrete syntax differs.

\paragraph{Power}
We cannot trivially make significant changes --- such as removing parentheses as mentioned earlier ---
to the code, because the lack of syntactic information renders us unable to easily judge whether
\begin{itemize}
  \item such a modification would alter the semantics of the code and
  \item such a modification would actually increase the readability of the code
\end{itemize}

\paragraph{Flexibility}
Information about constructs in the code and about the greater picture is often unavailable and
we have to deal with tight limitations on what we can safely change in the code without altering semantics.
This means that we cannot always just pick a preferred formatting style that we want to produce,
but instead have to accept compromises on the output style and focus on what we \textit{can} produce.
Allowing for further user configuration options of the desired output format would be borderline inconceivable.

\subsection{Uses}
This kind of formatter would not be suitable for some of the common use cases, including both
formatters integrated into an IDE or editor that run automatically or on key press and
formatters that provide a command-line interface (CLI) for use by developers.
Those will usually be applied to small or medium-sized source code files,
so the increased performance when compared to traditional formatters does not provide
a sufficiently large benefit to justify a trade-off in quality of the code style.

There are, however, some cases where the efficiency and potential streaming capability could
turn out to be useful for a more `on-the-fly' formatting.
Browser developer tools usually include a formatter for JavaScript files,
because those files are typically served as minified code that has
no spacing, no line breaks and consequentially no indentation at all.
Besides minification, the JavaScript preprocessing pipeline often also includes a bundling step
that merges all script and module files from the original source code into a single or very few
files in order to eliminate the need to serve each source file in a separate HTTP request.
Because of this common practice, the bundled files delivered to the browser can be quite large ---
sometimes multiple megabytes in size --- and the formatting process can take multiple seconds to complete.
Decreasing the total formatting time and the delay until the first lines are printed
could improve the developer experience in this use case,
while the `less pretty' formatting style would be tolerable,
since the code produced is usually looked at rather briefly and rarely modified.

We could also provide a CLI for the formatter and apply it to code files viewed in a terminal,
where a large delay before displaying the content would be unacceptable.
In particular, the formatter could be invoked on external code, perhaps downloaded or decompiled code.
The \code{less} pager utility supports file preprocessing by an arbitrary program
through the \code{LESSOPEN} environment variable; \autocite{lessMan}
this could be used for a formatter just as it is sometimes used for syntax highlighting
and it would greatly benefit from streaming-enabled formatting.
Some IDEs and editors can invoke an external program to process a file on keypress as well.

