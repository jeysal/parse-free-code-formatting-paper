\subsection{Indentation}
For the last aspect to implement, indentation,
we will use the regular nesting-based method,
not the possible aligning variant
we identified as an extension.

The first small change we need to make
stems from the realization that
decrementing the nesting level
because of a \code{rightPar} needs to happen
\textit{before} printing the token
and potentially inserting a
soft line break in front of it.
So we move the decrement up
to the beginning of the loop body
and then later on we merely
read and compare the nesting level:
\begin{minted}{javascript}
  // nesting decrement needs to happen before print
  if (type === 'rightPar') nestingLevel--;

  // most of the other code is here

  if (type === 'rightPar' && nestingLevel === 0) hardBreak();
\end{minted}

Now we can add the actual code for printing indentation spaces.
After calling \code{breakLine()} from \code{softBreak()},
we generate an indentation string with
a total of \code{nestingLevel * INDENT_SIZE} spaces.
Just like we defined \code{MAX_LINE_LENGTH} as 42,
we define \code{INDENT_SIZE} as 2 for further examples.
We append the indentation string to the code
and update the current \code{lineLength} to match:
\begin{minted}{javascript}
  const softBreak = () => {
    breakLine();

    const indentSize = nestingLevel * INDENT_SIZE;
    const indentation = Array(indentSize)
      .fill(' ')
      .join('');

    code += indentation;
    lineLength = indentSize;
  };
\end{minted}
We can now also move the assignment of 0 to \code{lineLength}
from \code{breakLine()} into \code{hardBreak()} ---
it would only get overwritten anyway.

\paragraph{Interaction with empty lines}
This indentation code works,
but has an awkward interaction with empty lines.
It is possible to generate code like this:
\begin{minted}{lisp}
; limit -|
(xxxxx (
    xxxxx

xxxxx
    xxx))
\end{minted}
An empty line only generates hard breaks,
so the third identifier in this example
is not indented at all.

We can avoid this by changing the
empty line insertion code from
\begin{minted}{javascript}
  while (consecutiveBreaks < 2) hardBreak();
\end{minted}
to
\begin{minted}{javascript}
  if (consecutiveBreaks === 0) hardBreak();
    softBreak();
\end{minted}
, so only the first break (if present at all) is a hard line break,
and the second break is a soft line break
that indents the next line correctly.

This indentation could be considered a violation of
the rule that the nesting level decreases \textit{before}
a \code{rightPar}, potentially printing something like
\begin{minted}{lisp}
; limit -|
(xxxxxxxxx
  )
\end{minted}
because we do not actually indent while printing the \code{rightPar},
but while printing the \code{emptyLine},
however this is a rare edge case
that is probably not worth introducing an extra check for.
