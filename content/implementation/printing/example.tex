\section{Formatting example}
If we run some Lisp-like code
based on arbitrary Lisp examples
\autocite{factorialExample}
\autocite{fibonacciExample}
\autocite{birthdayExample}
--- including the factorial example we used earlier ---
through the lexer and printer,
we get the results shown in Figure~\ref{fig:formattingExample}
on a separate page.

\begin{figure}[p]
  \begin{minted}{lisp}
    ; Input:
    ; factorial
    (defun factorial(n)(if(=n 0)1(*n(factorial(-n 1)))))

    ; fibonacci
    (defun fib(n)(if(lte n 2)1(+(fib(-n 1))(fib(-n 2)))))

    ; birthday paradox, already well-formatted
    (defconstant yearsize 365)
    (defun birthdayparadox (probability numberofpeople)
      (let ((newprobability (* (/
          (- yearsize numberofpeople) yearsize) probability)))
        (if (lt newprobability (/ 1 2))
            (1+ numberofpeople)
            (birthdayparadox newprobability
                             (1+ numberofpeople)))))

    ; Output:
    ; factorial
    (defun factorial (n) (if (= n 0) 1 (* n (
            factorial (- n 1)))))

    ; fibonacci
    (defun fib (n) (if (lte n 2) 1 (+ (fib (-
              n 1)) (fib (- n 2)))))

    ; birthday paradox, already well-formatted
    (defconstant yearsize 365)
    (defun birthdayparadox (probability
        numberofpeople) (let ((newprobability
            (* (/ (- yearsize numberofpeople)
                yearsize) probability))) (if (
            lt newprobability (/ 1 2)) (1 +
            numberofpeople) (birthdayparadox
            newprobability (1 + numberofpeople
            )))))
  \end{minted}
  \caption{Exemplary formatting input and
    output code}\label{fig:formattingExample}
\end{figure}
