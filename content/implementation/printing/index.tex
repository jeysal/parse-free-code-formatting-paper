\section{Printing}
Once our lexer has tokenized the input code,
we pass the array of tokens it generated into the printer.
The printer will then generate well-formatted output code from those tokens.
Its signature is the inverse of the lexer:
\begin{minted}{typescript}
  (tokens: {
    type: 'leftPar' | 'rightPar' | 'operator' | 'prefix' |
      'numLiteral' | 'boolLiteral' | 'keyword' | 'identifier' |
      'lineComment' | 'blockComment' | 'emptyLine',
    value: string,
  }[]) => string
\end{minted}

The high-level structure of the printer function is as follows:
\begin{minted}{javascript}
  const print = tokens => {
    let code = '';
    // GLOBAL_STATE ...

    for (const { type, value } of tokens) {
      // PRE_PRINT ...

      // print token
      code += value;

      // POST_PRINT ...
    }

    return code;
  };
\end{minted}
The printer iterates over all input tokens
and prints them to the output code.
Outside of that loop, it holds some global state,
most notably the output code generated so far.
As we gradually implement
more functionality for the aspects of printing
over the course of this section,
we will add more state variables
as well as computations and further printing
before and after a token is printed in the loop.

\subsection{Spacing}
% TODO standard spacing conditions that we already defined


\subsection{Hard line breaks}
We need to insert a hard line break
whenever a \code{rightPar} decreases the
current level of nesting to zero.
We track the level of nesting with another
state variable declared outside of the printer loop:
\begin{minted}{javascript}
  let nestingLevel = 0;
\end{minted}

In the \code{POST_PRINT} position,
after the assignment of \code{prevAllowsSpace},
we track changes to the nesting level and
increment it after a \code{leftPar}:
\begin{minted}{javascript}
  if (type === 'leftPar') nestingLevel++;
\end{minted}

For \code{rightPar}s, we need to
not only decrement the nesting level,
but also print a line break if
it has reached zero because of the decrement.
This line break must also effect our spacing,
making sure that no space is printed
in addition to the line break
in the next loop iteration.
We achieve this with the following code:
\begin{minted}{javascript}
  if (type === 'rightPar') {
    if (--nestingLevel === 0) {
      code += '\n';
      prevAllowsSpace = false; // line break replaces space
    }
  }
\end{minted}

\paragraph{Preservation of empty lines}
% TODO additions to printer that preserve empty lines


\subsection{Soft line breaks}
% TODO greedy line breaks with separation avoidance and inevitable length violation recovery


\subsection{Indentation}
% TODO regular indentation, not the aligning extension
% TODO move decrement of nestingLevel to PRE_PRINT, as required by 4.3.2 Indentation


\section{Formatting examples}
% TODO code examples both from earlier in the paper and from external lisp code

