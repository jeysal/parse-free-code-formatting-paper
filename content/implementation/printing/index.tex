\section{Printing}
Analogous to the lexer, we implement the printer by
defining its signature and basic structure
and then successively enriching the code
with logic that for each aspect of printing
that we need to tackle.
When the final bit of printing logic is implemented,
we can look at some examples of code
that is passed through the finished
lexer and printer functions for reformatting.

Once our lexer has tokenized the input code,
the array of tokens it generated is passed into the printer.
The printer will then generate well-formatted output code from those tokens.
Its signature is the inverse of the lexer:
\begin{minted}{typescript}
  (tokens: {
    type: 'leftPar' | 'rightPar' | 'operator' | 'prefix' |
      'numLiteral' | 'boolLiteral' | 'keyword' | 'identifier' |
      'lineComment' | 'blockComment' | 'emptyLine',
    value: string,
  }[]) => string
\end{minted}

The high-level structure of the printer function is as follows:
\begin{minted}{javascript}
  const print = tokens => {
    let code = '';
    // GLOBAL_STATE ...

    for (const { type, value } of tokens) {
      // PRE_PRINT ...

      // print token
      code += value;

      // POST_PRINT ...
    }

    return code;
  };
\end{minted}
The printer iterates over all input tokens
and prints them to the output code.
Outside of that loop, it holds some global state,
most notably the output code generated so far.
As we gradually implement
more functionality for the aspects of printing
over the course of this section,
we will add more state variables
as well as computations and further printing
before and after a token is printed in the loop.

\subsection{Spacing}
% TODO standard spacing conditions that we already defined


\subsection{Hard line breaks}
We need to insert a hard line break
whenever a \code{rightPar} decreases the
current level of nesting to zero.
We track the level of nesting with another
state variable declared outside of the printer loop:
\begin{minted}{javascript}
  let nestingLevel = 0;
\end{minted}

In the \code{POST_PRINT} position,
after the assignment of \code{prevAllowsSpace},
we track changes to the nesting level and
increment it after a \code{leftPar}:
\begin{minted}{javascript}
  if (type === 'leftPar') nestingLevel++;
\end{minted}

For \code{rightPar}s, we need to
not only decrement the nesting level,
but also print a line break if
it has reached zero because of the decrement.
This line break must also effect our spacing,
making sure that no space is printed
in addition to the line break
in the next loop iteration.
We achieve this with the following code:
\begin{minted}{javascript}
  if (type === 'rightPar') {
    if (--nestingLevel === 0) {
      code += '\n';
      prevAllowsSpace = false; // line break replaces space
    }
  }
\end{minted}

\paragraph{Preservation of empty lines}
When discussing the hard line breaks for our Lisp-like language,
we also noticed that it is important
to preserve empty lines that were present in the input code,
because the author of the code will have inserted those
to aid reader comprehension of the code structure.
We have built functionality in our lexer specifically
to insert tokens where empty lines occur;
now we need to adapt the printer to use those tokens
in order to generate empty lines again.

An empty line can be inserted by
printing two consecutive line break characters.
To keep track of how many of those we have printed
we introduce another state variable:
\begin{minted}{javascript}
  let consecutiveBreaks = 2;
\end{minted}
We will use this variable to ensure that we never print
more than two consecutive line break characters,
because we do not want
more than one consecutive empty line
in our output code.
We initialize it with 2,
pretending that we have already printed an empty line
at the beginning of the code,
because we do not want to print an actual empty line
before a meaningful character at the start.

When we encounter an \code{emptyLine} token in the printer loop,
between the assignment of \code{prevAllowsSpace}
and the parentheses handling,
we count the variable \code{consecutiveBreaks} up to 2
and print a line break for each increment:
\begin{minted}{javascript}
  if (type === 'emptyLine') {
    while (consecutiveBreaks < 2) {
      code += '\n';
      consecutiveBreaks++;
      prevAllowsSpace = false; // line break replaces space
    }
  }
\end{minted}
For any other token,
we reset \code{consecutiveBreaks} to 0,
because we know that we have stopped the series of line breaks
by printing the value of the current token in this iteration:
\begin{minted}{javascript}
  if (type === 'emptyLine') {
    // ...
  } else {
    consecutiveBreaks = 0;
  }
\end{minted}

To avoid printing two empty lines if an \code{emptyLine} token
occurs after a \code{rightPar} token
that causes a regular hard break insertion,
we also need to add the statement \code{consecutiveBreaks++;}
in the code block where we print that kind of line break.


\subsection{Soft line breaks}
Next, we add soft line breaking capabilities to our formatter.
We use the greedy line-filling algorithm,
as it operates in a simple, linear way
that can be implemented in our formatter without
adding significantly to its code complexity.
We will also employ our strategy for
gracefully handling inevitable length limit violations,
but will omit the separation avoidance,
which would require more fundamental changes
to the printer implementation.

Let us take a look at what our printer currently looks like.
With a few simplifications applied,
notably the extraction of the hard line break insertion logic
that we implemented into its own function, \code{breakLine()},
the code we have at this point is shown
in figure~\ref{fig:printerBeforeSoftLineBreaks} on a separate page.

\begin{figure}[p]
  \begin{minted}{javascript}
  const print = tokens => {
    let code = '';
    let nestingLevel = 0;
    let prevAllowsSpace = false;
    let consecutiveBreaks = 2;

    // helpers
    const breakLine = () => {
      code += '\n';
      consecutiveBreaks++;
      prevAllowsSpace = false; // line break replaces space
    };

    // main loop
    for (const { type, value } of tokens) {
      if (prevAllowsSpace && allowsSpaceBefore(type))
        code += ' ';

      // print token
      code += value;

      // set previous information for next iteration
      prevAllowsSpace = allowsSpaceAfter(type);

      // hard line break
      if (type === 'emptyLine')
        while (consecutiveBreaks < 2)
          breakLine();
      else consecutiveBreaks = 0;

      if (type === 'leftPar')
        nestingLevel++;
      if (type === 'rightPar' && --nestingLevel === 0)
        breakLine();
      if (type === 'lineComment')
        breakLine();
    }

    return code;
  };
  \end{minted}
  \caption{The printer code before implementation of
    soft line breaking.}\label{fig:printerBeforeSoftLineBreaks}
\end{figure}

We declare a variable to track the amount of characters
we printed since the last line break outside of the loop:
\begin{minted}{javascript}
  let lineLength = 0;
\end{minted}
In \code{breakLine}, we reassign it to 0
so we can start over when we enter a new line.
We also declare a length limit that we will try to never exceed.
This limit could be set by configuration options;
we will assume a length limit of 42 for our purposes:
\begin{minted}{javascript}
  const MAX_LINE_LENGTH = 42;
\end{minted}

Another change that we need to make to the current code
concerns the spacing information in the first statement of the loop.
We need to know whether a space will be printed before the token,
because whitespace also counts towards the line length.
We extract the condition into a variable,
and leave some space before the space printing
to insert our soft line breaking logic into:
\begin{minted}{javascript}
  let spaceBefore = prevAllowsSpace && allowsSpaceBefore(type);

  // soft line breaks

  if(spaceBefore)
    code += value;
\end{minted}

In the most common and simple case,
we can just go ahead and print our token
on the current line,
perhaps with a preceding space.
Then we will simply add the length of the token value
to the current \code{lineLength},
possibly adding another 1 on top if
the token will be printed with a space in front of it.
\begin{minted}{javascript}
  let printLength = value.length + spaceBefore;

  // what if the token does not fit?

  lineLength += printLength;
\end{minted}
On a side note, adding the boolean \code{spaceBefore}
in this code snippet is merely a shorter way of
writing \code{+ (spaceBefore ? 1 : 0)}.

But if the token does not fit on the same line anymore
without exceeding the length limit,
we will need to print a line break first:
\begin{minted}{javascript}
  if (lineLength + printLength > MAX_LINE_LENGTH) {
    breakLine();

    // No space after all, just the break
    if (spaceBefore) printLength--;
    spaceBefore = false;
  }
\end{minted}
In that case, we also prohibit printing a space before the current,
because the line break takes its place instead.
If necessary, the space is also subtracted back from the length
that we determined for the current token,
so we do not start the new line with an off-by-one \code{lineLength} value.

\paragraph{Handling length limit violation}
If a very long token or the indentation
that we will introduce in our printer next
force us to print beyond the line length limit,
we decided that we want to temporarily increase
the limit to the length of the violating line
until the next hard line break.
This helps mitigate the effect of the violation,
especially if it was caused by indentation,
which is likely to remain deep
for further lines we need to print.

To implement this mechanism,
we have to start distinguishing between
hard and soft line breaks in the printer.
We create two functions
\code{softBreak} and \code{hardBreak}
that both delegate to \code{breakLine}:
\begin{minted}{javascript}
  const softBreak = () => {
    breakLine();
  };
  const hardBreak = () => {
    breakLine();
  };
\end{minted}
The \code{breakLine} call we just added
becomes \code{softBreak}, while
the calls in the conditional statements
that handle \code{emptyLine}s, \code{rightPar}s
and \code{lineComment}s become \code{hardBreak}s.

To track the current, possibly increased, line length limit
until the next hard break, we use a variable initialized with
\begin{minted}{javascript}
  let maxLineLength = MAX_LINE_LENGTH;
\end{minted}
and reset to the same constant in the \code{hardBreak} function.
We replace the usage of the constant in the condition
that we introduced for deciding whether to insert a soft line break
by a usage of this variable.

To make sure that the variable is increased
after we printed a token beyond the limit,
we reassign it to the current line length or itself,
whichever is higher, after the soft line breaking code:
\begin{minted}{javascript}
  maxLineLength = Math.max(lineLength, maxLineLength);
\end{minted}


\subsection{Indentation}
For the last aspect to implement, indentation,
we will use the regular nesting-based method,
not the possible aligning variant
we identified as an extension.

The first small change we need to make
stems from the realization that
decrementing the nesting level
because of a \code{rightPar} needs to happen
\textit{before} printing the token
and potentially inserting a
soft line break in front of it.
So we move the decrement up
to the beginning of the loop body
and then later on we merely
read and compare the nesting level:
\begin{minted}{javascript}
  // nesting decrement needs to happen before print
  if (type === 'rightPar') nestingLevel--;

  // rest of the loop

  if (type === 'rightPar' && nestingLevel === 0) hardBreak();
\end{minted}

Now we can add the actual code for printing indentation spaces.
After calling \code{breakLine()} from \code{softBreak()},
we generate an indentation string with
a total of \code{nestingLevel * INDENT_SIZE} spaces.
Just like we defined \code{MAX_LINE_LENGTH} as 42,
we define \code{INDENT_SIZE} as 2 for further examples.
We append the indentation string to the code
and update the current \code{lineLength} to match:
\begin{minted}{javascript}
  const softBreak = () => {
    breakLine();

    const indentSize = nestingLevel * INDENT_SIZE;
    const indentation = Array(indentSize)
      .fill(' ')
      .join('');

    code += indentation;
    lineLength = indentSize;
  };
\end{minted}
We can now also move the assignment of 0 to \code{lineLength}
from \code{breakLine()} into \code{hardBreak()} ---
it would only get overwritten anyway.

\paragraph{Interaction with empty lines}
This indentation code works,
but has an awkward interaction with empty lines.
It is possible to generate code like this:
\begin{minted}{lisp}
; limit -|
(xxxxx (
    xxxxx

xxxxx
    xxx))
\end{minted}
An empty line only generates hard breaks,
so the third identifier in this example
is not indented at all.

We can avoid this by changing the
empty line insertion code from
\begin{minted}{javascript}
  while (consecutiveBreaks < 2) breakLine();
\end{minted}
to
\begin{minted}{javascript}
  if (consecutiveBreaks === 0) hardBreak();
    softBreak();
\end{minted}
, so only the first break (if present at all) is a hard line break,
and the second break is a soft line break
that indents the next line correctly.

This indentation could be considered a violation of
the rule that the nesting level decreases \textit{before}
a \code{rightPar}, potentially printing something like
\begin{minted}{lisp}
; limit -|
(xxxxxxxxx
  )
\end{minted}
because we do not actually indent while printing the \code{rightPar},
but while printing the \code{emptyLine},
however this is a rare edge case
that is probably not worth introducing an extra check for.


\section{Formatting example}
If we run some Lisp-like code
based on arbitrary Lisp examples
\autocite{factorialExample}
\autocite{fibonacciExample}
\autocite{birthdayExample}
--- including the factorial example we used earlier ---
through the lexer and printer,
we get the results shown in Figure~\ref{fig:formattingExample}
on a separate page.

\begin{figure}[p]
  \begin{minted}{lisp}
    ; Input:
    ; factorial
    (defun factorial(n)(if(=n 0)1(*n(factorial(-n 1)))))

    ; fibonacci
    (defun fib(n)(if(lte n 2)1(+(fib(-n 1))(fib(-n 2)))))

    ; birthday paradox, already well-formatted
    (defconstant yearsize 365)
    (defun birthdayparadox (probability numberofpeople)
      (let ((newprobability (* (/
          (- yearsize numberofpeople) yearsize) probability)))
        (if (lt newprobability (/ 1 2))
            (1+ numberofpeople)
            (birthdayparadox newprobability
                             (1+ numberofpeople)))))

    ; Output:
    ; factorial
    (defun factorial (n) (if (= n 0) 1 (* n (
            factorial (- n 1)))))

    ; fibonacci
    (defun fib (n) (if (lte n 2) 1 (+ (fib (-
              n 1)) (fib (- n 2)))))

    ; birthday paradox, already well-formatted
    (defconstant yearsize 365)
    (defun birthdayparadox (probability
        numberofpeople) (let ((newprobability
            (* (/ (- yearsize numberofpeople)
                yearsize) probability))) (if (
            lt newprobability (/ 1 2)) (1 +
            numberofpeople) (birthdayparadox
            newprobability (1 + numberofpeople
            )))))
  \end{minted}
  \caption{Exemplary formatting input and
    output code}\label{fig:formattingExample}
\end{figure}

