\section{Printing}
Once our lexer has tokenized the input code,
we pass the array of tokens it generated into the printer.
The printer will then generate well-formatted output code from those tokens.
Its signature is the inverse of the lexer:
\begin{minted}{typescript}
  (tokens: {
    type: 'leftPar' | 'rightPar' | 'operator' | 'prefix' |
      'numLiteral' | 'boolLiteral' | 'keyword' | 'identifier' |
      'lineComment' | 'blockComment' | 'emptyLine',
    value: string,
  }[]) => string
\end{minted}

The high-level structure of the printer function is as follows:
\begin{minted}{javascript}
  const print = tokens => {
    let code = '';
    // GLOBAL_STATE ...

    for (const { type, value } of tokens) {
      // PRE_PRINT ...

      // print token
      code += value;

      // POST_PRINT ...
    }

    return code;
  };
\end{minted}
The printer iterates over all input tokens
and prints them to the output code.
Outside of that loop, it holds some global state,
most notably the output code generated so far.
As we gradually implement
more functionality for the aspects of printing
over the course of this section,
we will add more state variables
as well as computations and further printing
before and after a token is printed in the loop.

\subsection{Spacing}
% TODO standard spacing conditions that we already defined


\subsection{Hard line breaks}
We need to insert a hard line break
whenever a \code{rightPar} decreases the
current level of nesting to zero.
We track the level of nesting with another
state variable declared outside of the printer loop:
\begin{minted}{javascript}
  let nestingLevel = 0;
\end{minted}

In the \code{POST_PRINT} position,
after the assignment of \code{prevAllowsSpace},
we track changes to the nesting level and
increment it after a \code{leftPar}:
\begin{minted}{javascript}
  if (type === 'leftPar') nestingLevel++;
\end{minted}

For \code{rightPar}s, we need to
not only decrement the nesting level,
but also print a line break if
it has reached zero because of the decrement.
This line break must also effect our spacing,
making sure that no space is printed
in addition to the line break
in the next loop iteration.
We achieve this with the following code:
\begin{minted}{javascript}
  if (type === 'rightPar') {
    if (--nestingLevel === 0) {
      code += '\n';
      prevAllowsSpace = false; // line break replaces space
    }
  }
\end{minted}

\paragraph{Preservation of empty lines}
When discussing the hard line breaks for our Lisp-like language,
we also noticed that it is important
to preserve empty lines that were present in the input code,
because the author of the code will have inserted those
to aid reader comprehension of the code structure.
We have built functionality in our lexer specifically
to insert tokens where empty lines occur;
now we need to adapt the printer to use those tokens
in order to generate empty lines again.

An empty line can be inserted by
printing two consecutive line break characters.
To keep track of how many of those we have printed
we introduce another state variable:
\begin{minted}{javascript}
  let consecutiveBreaks = 2;
\end{minted}
We will use this variable to ensure that we never print
more than two consecutive line break characters,
because we do not want
more than one consecutive empty line
in our output code.
We initialize it with 2,
pretending that we have already printed an empty line
at the beginning of the code,
because we do not want to print an actual empty line
before a meaningful character at the start.

When we encounter an \code{emptyLine} token in the printer loop,
between the assignment of \code{prevAllowsSpace}
and the parentheses handling,
we count the variable \code{consecutiveBreaks} up to 2
and print a line break for each increment:
\begin{minted}{javascript}
  if (type === 'emptyLine') {
    while (consecutiveBreaks < 2) {
      code += '\n';
      consecutiveBreaks++;
      prevAllowsSpace = false; // line break replaces space
    }
  }
\end{minted}
For any other token,
we reset \code{consecutiveBreaks} to 0,
because we know that we have stopped the series of line breaks
by printing the value of the current token in this iteration:
\begin{minted}{javascript}
  if (type === 'emptyLine') {
    // ...
  } else {
    consecutiveBreaks = 0;
  }
\end{minted}

To avoid printing two empty lines if an \code{emptyLine} token
occurs after a \code{rightPar} token
that causes a regular hard break insertion,
we also need to add the statement \code{consecutiveBreaks++;}
in the code block where we print that kind of line break.


\subsection{Soft line breaks \& indentation}
% TODO regular indentation, not the aligning extension
% TODO greedy line breaks with separation avoidance and inevitable length violation recovery
% TODO possible to split this up and do indentation strictly after soft line breaks?


\section{Formatting examples}
% TODO code examples both from earlier in the paper and from external lisp code

