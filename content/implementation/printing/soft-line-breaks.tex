\subsection{Soft line breaks}
Next, we add soft line breaking capabilities to our formatter.
We use the greedy line-filling algorithm,
as it operates in a simple, linear way
that can be implemented in our formatter without
adding significantly to its code complexity.
We will also employ our strategy for
gracefully handling inevitable length limit violations,
but will omit the separation avoidance,
which would require more fundamental changes
to the printer implementation.

Let us take a look at what our printer currently looks like.
With a few simplifications applied,
notably the extraction of the hard line break insertion logic
that we implemented into its own function, \code{breakLine()},
the code we have at this point is shown
in figure~\ref{fig:printerBeforeSoftLineBreaks} on a separate page.

\begin{figure}[p]
  \begin{minted}{javascript}
  const print = tokens => {
    let code = '';
    let nestingLevel = 0;
    let prevAllowsSpace = false;
    let consecutiveBreaks = 2;

    // helpers
    const breakLine = () => {
      code += '\n';
      consecutiveBreaks++;
      prevAllowsSpace = false; // line break replaces space
    };

    // main loop
    for (const { type, value } of tokens) {
      if (prevAllowsSpace && allowsSpaceBefore(type))
        code += ' ';

      // print token
      code += value;

      // set previous information for next iteration
      prevAllowsSpace = allowsSpaceAfter(type);

      // hard line break
      if (type === 'emptyLine')
        while (consecutiveBreaks < 2)
          breakLine();
      else consecutiveBreaks = 0;

      if (type === 'leftPar')
        nestingLevel++;
      if (type === 'rightPar' && --nestingLevel === 0)
        breakLine();
      if (type === 'lineComment')
        breakLine();
    }

    return code;
  };
  \end{minted}
  \caption{The printer code before implementation of
    soft line breaking.}\label{fig:printerBeforeSoftLineBreaks}
\end{figure}

We declare a variable to track the amount of characters
we printed since the last line break outside of the loop:
\begin{minted}{javascript}
  let lineLength = 0;
\end{minted}
In \code{breakLine}, we reassign it to 0
so we can start over when we enter a new line.
We also declare a length limit that we will try to never exceed.
This limit could be set by configuration options;
we will assume a length limit of 42 for our purposes:
\begin{minted}{javascript}
  const MAX_LINE_LENGTH = 42;
\end{minted}

Another change that we need to make to the current code
concerns the spacing information in the first statement of the loop.
We need to know whether a space will be printed before the token,
because whitespace also counts towards the line length.
We extract the condition into a variable,
and leave some space before the space printing
to insert our soft line breaking logic into:
\begin{minted}{javascript}
  let spaceBefore = prevAllowsSpace && allowsSpaceBefore(type);

  // soft line breaks

  if(spaceBefore)
    code += value;
\end{minted}

In the most common and simple case,
we can just go ahead and print our token
on the current line,
perhaps with a preceding space.
Then we will simply add the length of the token value
to the current \code{lineLength},
possibly adding another 1 on top if
the token will be printed with a space in front of it.
\begin{minted}{javascript}
  let printLength = value.length + spaceBefore;

  // what if the token does not fit?

  lineLength += printLength;
\end{minted}
On a side note, adding the boolean \code{spaceBefore}
in this code snippet is merely a shorter way of
writing \code{+ (spaceBefore ? 1 : 0)}.

But if the token does not fit on the same line anymore
without exceeding the length limit,
we will need to print a line break first:
\begin{minted}{javascript}
  if (lineLength + printLength > MAX_LINE_LENGTH) {
    breakLine();

    // No space after all, just the break
    if (spaceBefore) printLength--;
    spaceBefore = false;
  }
\end{minted}
In that case, we also prohibit printing a space before the current,
because the line break takes its place instead.
If necessary, the space is also subtracted back from the length
that we determined for the current token,
so we do not start the new line with an off-by-one \code{lineLength} value.

\paragraph{Handling length limit violation}
If a very long token or the indentation
that we will introduce in our printer next
force us to print beyond the line length limit,
we decided that we want to temporarily increase
the limit to the length of the violating line
until the next hard line break.
This helps mitigate the effect of the violation,
especially if it was caused by indentation,
which is likely to remain deep
for further lines we need to print.

To implement this mechanism,
we have to start distinguishing between
hard and soft line breaks in the printer.
We create two functions
\code{softBreak} and \code{hardBreak}
that both delegate to \code{breakLine}:
\begin{minted}{javascript}
  const softBreak = () => {
    breakLine();
  };
  const hardBreak = () => {
    breakLine();
  };
\end{minted}
The \code{breakLine} call we just added
becomes \code{softBreak}, while
the calls in the conditional statements
that handle \code{emptyLine}s, \code{rightPar}s
and \code{lineComment}s become \code{hardBreak}s.

To track the current, possibly increased, line length limit
until the next hard break, we use a variable initialized with
\begin{minted}{javascript}
  let maxLineLength = MAX_LINE_LENGTH;
\end{minted}
and reset to the same constant in the \code{hardBreak} function.
We replace the usage of the constant in the condition
that we introduced for deciding whether to insert a soft line break
by a usage of this variable.

To make sure that the variable is increased
after we printed a token beyond the limit,
we reassign it to the current line length or itself,
whichever is higher, after the soft line breaking code:
\begin{minted}{javascript}
  maxLineLength = Math.max(lineLength, maxLineLength);
\end{minted}
