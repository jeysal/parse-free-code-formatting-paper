\subsection{Structure}
The lexer function holds two essential values:
\begin{itemize}
  \item the current position in the source code: \mintinline{javascript}{let position = 0;}
  \item a mutable array of the tokens recognized so far: \mintinline{javascript}{const tokens = [];}
\end{itemize}
The return value in the success case is the \code{tokens} array.

The backbone of the lexing process is formed by the loop:
\begin{minted}{javascript}
  while (position < code.length) {
    let char = code[position];

    // ... token detection here ...

    throw new Error(
      `unexpected character ${char} at position ${position}`);
  }
\end{minted}
It runs until the position cursor reaches the end of the input code
and reads the current character into the \code{char} variable in every iteration.
We can then insert arbitrary token detection code that pushes new tokens to the \code{tokens} array
into the loop where the placeholder comment is positioned.
This code is also responsible for advancing the \code{position} cursor
and triggering the next iteration using \code{continue;} after doing its work.
If the token detection code did not successfully match a token,
we have found an illegal character in the source code and throw an error.
