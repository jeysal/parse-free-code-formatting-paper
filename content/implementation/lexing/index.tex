\section{Lexing}
Our formatter is built to do without parsing any code,
but we still need to split the input code into individual tokens.
Therefore, we will start by creating a lexer that can recognize
all the token types we defined for our lisp-like language,
including line and block comments from the comments extension.
We could use a parser generator such as ANTLR to generate a lexer only,
but due to the simple nature of our language,
a lexer will be sufficiently trivial to implement ourselves.

The lexer function has the signature
\begin{minted}{typescript}
(code: string) => {
  type: 'leftPar' | 'rightPar' | 'operator' | 'prefix' |
    'numLiteral' | 'boolLiteral' | 'keyword' | 'identifier' |
    'lineComment' | 'blockComment',
  value: string,
}[]
\end{minted}
, taking the input code as a string and returning an array of
token objects with a property \code{type} from a fixed pool of
strings that contains a type identifier
for each possible token that can occur in the code,
and a a property \code{value} that contains the
snippet from the input code that represents the corresponding token.
The token types are exactly those we defined in the language grammar,
plus \code{'lineComment'} and \code{'blockComment'}
for representing the comments we defined as an extension to the language,
which have no effect on the code semantics,
but are treated like any other token for the purpose of formatting.

\subsection{Examples}

% TODO lexer test cases


\subsection{Structure}

% TODO function, vars
% TODO while loop, error case


\subsection{Whitespace}

% TODO skip whitespace


\subsection{Parentheses}

% TODO detect parentheses


\subsection{Operators \& prefixes}
Operators are single-character tokens consisting of one of the characters:
\begin{minted}{javascript}
  const operators = ['=', '+', '-', '*', '/'];
\end{minted}

If the current character is one of those in the \code{operators} array,
we push a new token of type \code{operator}:
\begin{minted}{javascript}
  if (operators.includes(char)) {
    tokens.push({ type: 'operator', value: char });
    position++;
    continue;
  }
\end{minted}

For prefixes, we do the same based on the characters:
\begin{minted}{javascript}
  const prefixes = ["'", '&'];
\end{minted}


\subsection{Number literals}

% TODO detect number literals


\subsection{Identifiers, keywords \& boolean literals}

% TODO detect identifiers, keywords & boolean literals


\subsection{Line comments}
For detecting line comments, we use the regular loop structure.
The initial detection checks for the semicolon character:
\begin{minted}{javascript}
  if (char === ';') {
    // ...
  }
\end{minted}

The loop runs until the end of the line by checking for line breaks:
\begin{minted}{javascript}
  do {
    // ...
  } while (char !== '\n');
\end{minted}
Line break characters may vary depending by operating system,
but we will assume \code{\n} for the sake of simplicity
and stay focused on the actual formatting algorithms.

After the loop, we push a token as usual and continue with the main loop.


\subsection{Block comments}
The final block of code for lexing detects block comments.
Those are delimited by two characters (\code{#| ... |#}),
so we perform two checks before entering the loop:
\begin{minted}{javascript}
  if (char === '#') {
    let commentChar = code[++position];
    if (commentChar === '|') {
      // ...
    }
  }
\end{minted}
If the inner check fails, we will fall through to the throw statement
at the bottom of the lexer main loop and report that
the character after the \code{#} is invalid.

If both conditions are truthy, we have detected a block comment.
We initialize the variable to store its text
with the character \code{#} instead of an empty string,
because we have already advanced the cursor beyond its occurrence in the code,
so the loop will start at the pipe character (\code{|}):
\begin{minted}{javascript}
  let text = '#';
\end{minted}

The exit condition for the loop is the detection
of the block comment terminator (\code{|#}),
which will be the last bit of code included in the text of the token:
\begin{minted}{javascript}
  do {
    text += commentChar;
    commentChar = code[++position];
  } while (!text.endsWith('|#'));
\end{minted}

The block ends by pushing a new token and continuing the main loop:
\begin{minted}{javascript}
  tokens.push({ type: 'blockComment', value: text });
  continue;
\end{minted}

