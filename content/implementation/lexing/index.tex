\section{Lexing}
Our formatter is built to do without parsing any code,
but we still need to split the input code into individual tokens.
Therefore, we will start by creating a lexer that can recognize
all the token types we defined for our lisp-like language,
including line and block comments from the comments extension.
We could use a parser generator such as ANTLR to generate a lexer only,
but due to the simple nature of our language,
a lexer will be sufficiently trivial to implement ourselves.

The lexer function has the signature
\begin{minted}{typescript}
  (code: string) => {
    type: 'leftPar' | 'rightPar' | 'operator' | 'prefix' |
      'numLiteral' | 'boolLiteral' | 'keyword' | 'identifier' |
      'lineComment' | 'blockComment' | 'emptyLine',
    value: string,
  }[]
\end{minted}
, taking the input code as a string and returning an array of
token objects with a property \code{type} from a fixed pool of
strings that contains a type identifier
for each possible token that can occur in the code,
and a a property \code{value} that contains the
snippet from the input code that represents the corresponding token.
The token types are exactly those we defined in the language grammar,
plus \code{'lineComment'} and \code{'blockComment'}
for representing the comments we defined as an extension to the language,
which have no effect on the code semantics,
but are treated like any other token for the purpose of formatting,
and \code{'emptyLine'} for preserving empty lines in the code,
which we deemed essential to the hard breaking algorithm of a formatter.

\subsection{Examples}
The following example input code strings and corresponding expected output token arrays
demonstrate the intended functionality that the lexer function should provide
and can be used as test cases to verify that a lexer implementation works as expected.

\paragraph{Regular token types}
The lexer recognizes the regular token types
as defined by the language grammar
and ignores whitespace inbetween them:
\begin{minted}{lisp}
  (lambda id1 &id2
    (= true 123))
\end{minted}
is tokenized to:
\begin{minted}{javascript}
  [
    { type: 'leftPar', value: '(' },
    { type: 'keyword', value: 'lambda' },
    { type: 'identifier', value: 'id1' },
    { type: 'prefix', value: '&' },
    { type: 'identifier', value: 'id2' },
    { type: 'leftPar', value: '(' },
    { type: 'operator', value: '=' },
    { type: 'boolLiteral', value: 'true' },
    { type: 'numLiteral', value: '123' },
    { type: 'rightPar', value: ')' },
    { type: 'rightPar', value: ')' },
  ]
\end{minted}

\paragraph{Line comments}
The lexer recognizes line comments
as used in the comments extension:
\begin{minted}{lisp}
  (a ;comment
  b)
\end{minted}
is tokenized to:
\begin{minted}{javascript}
  [
    { type: 'leftPar', value: '(' },
    { type: 'identifier', value: 'a' },
    { type: 'lineComment', value: ';comment' },
    { type: 'identifier', value: 'b' },
    { type: 'rightPar', value: ')' },
  ]
\end{minted}

\paragraph{Block comments}
The lexer recognizes block comments
as used in the comments extension:
\begin{minted}{lisp}
  (a #|com
  ment|# b)
\end{minted}
is tokenized to:
\begin{minted}{javascript}
  [
    { type: 'leftPar', value: '(' },
    { type: 'identifier', value: 'a' },
    { type: 'blockComment', value: '#|com\nment|#' },
    { type: 'identifier', value: 'b' },
    { type: 'rightPar', value: ')' },
  ]
\end{minted}

\paragraph{Illegal characters}
The lexer throws an error
if the code includes an illegal character:
\begin{minted}{text}
  (a # b)
\end{minted}
results in a lexer error.


\subsection{Structure}
The lexer function holds two essential values:
\begin{itemize}
  \item the current position in the source code: \mintinline{javascript}{let position = 0;}
  \item a mutable array of the tokens recognized so far: \mintinline{javascript}{const tokens = [];}
\end{itemize}
The return value in the success case is the \code{tokens} array.

The backbone of the lexing process is formed by the loop:
\begin{minted}{javascript}
  while (position < code.length) {
    let char = code[position];

    // ... token detection here ...

    throw new Error(
      `unexpected character ${char} at position ${position}`);
  }
\end{minted}
It runs until the position cursor reaches the end of the input code
and reads the current character into the \code{char} variable in every iteration.
We can then insert arbitrary token detection code that pushes new tokens to the \code{tokens} array
into the loop where the placeholder comment is positioned.
This code is also responsible for advancing the \code{position} cursor
and triggering the next iteration using \code{continue;} after doing its work.
If the token detection code did not successfully match a token,
we have found an illegal character in the source code and throw an error.


\subsection{Whitespace}
Whitespace primarily includes spaces, tabs, line feeds and carriage returns,
all of which are characters which will not be included in any token
but may serve as separators between tokens,
for example two identifiers that would be treated as one
if there were no whitespace between them.

It can be matched with the regular expression:
\begin{minted}{javascript}
  const whitespaceRegex = /\s/;
\end{minted}

We use that regular expression to skip all of it:
\begin{minted}{javascript}
  if (whitespaceRegex.test(char)) {
    position++;
    continue;
  }
\end{minted}


\subsection{Parentheses}
Parentheses are single-character tokens that can have exactly one value ---
\code{(} for \code{leftPar}s and \code{)} for \code{rightPar}s.

We push a new token whenever we detect one of them with
\begin{minted}{javascript}
  if (char === '(') {
    tokens.push({ type: 'leftPar', value: char });
    position++;
    continue;
  }
\end{minted}
and the equivalent for \code{rightPar}s afterwards.


\subsection{Operators \& prefixes}

% TODO detect operators & prefixes


\subsection{Number literals}
Number literals are a little more complex to recognize,
because they are the first token so far to have variable length.

Initial detection is simple using the a regex that matches digits:
\begin{minted}{javascript}
  const numRegex = /[0-9]/;

  if (numRegex.test(char)) {
    // ...
  }
\end{minted}

But after the initial digit, the number literal may contain further digits.
We use a loop to collect them in a string,
running until we find a character that is not a digit:
\begin{minted}{javascript}
  let numberLiteral = '';

  do {
    numberLiteral += char;
    char = code[++position];
  } while (numRegex.test(char));
\end{minted}

Afterwards, we can push the token as usual,
with the value set to the string of accumulated digits:
\begin{minted}{javascript}
  tokens.push({ type: 'numLiteral', value: numberLiteral });
  continue;
\end{minted}
The \code{position} cursor has already been advanced
beyond the number literal in the last iteration of the loop,
so we do not need to increment it like we did in the other blocks.

This also implies that the lexer will accept any token
immediately after the number literal --- the code
\begin{minted}{lisp}
  (123abc)
\end{minted}
would be tokenized into the number literal \code{123}
and, by a rule we have yet to implement, the identifier \code{abc}.
The language might not allow this pattern to occur in the code
without any separating whitespace, so we may accept code that is not valid ---
and in this concrete case later reformat it to valid code that
conforms to what was likely the intended code in the first place:
\begin{minted}{lisp}
  (123 abc)
\end{minted}

In any case, accepting a superset of code that is actually valid
by the rules of the language is not something we can avoid,
because the lack of parsing means that
we can only check the validity of the input
according to the lexical grammar,
but not according to the syntactic grammar.
We can tolerate these instances of reformatting code that is invalid,
because our printing algorithm will never introduce significant changes
that destroy the code further and it will recover a well-formatted state
after the user fixes the code error and reformats once again.


\subsection{Identifiers, keywords \& boolean literals}
Identifiers, keywords and boolean literals require
the same loop structure as number literals.
Instead of testing with the \code{numRegex}, we test with
\begin{minted}{javascript}
  const alphaRegex = /[a-zA-Z]/;
\end{minted}
for the first character in the if statement, and
\begin{minted}{javascript}
  const alphaNumRegex = /[a-zA-Z0-9]/;
\end{minted}
for the remaining characters in the do-while statement and store them all in
\begin{minted}{javascript}
  let name = '';
\end{minted}

Only once the loop has completed,
we can use the knowledge about all keywords and boolean literals that exist
\begin{minted}{javascript}
  const booleans = ['true', 'false'];
  const keywords = ['quote', 'lambda', 'defun',
                    'let', 'if', 'and', 'or'];
\end{minted}
to determine whether the token we found belongs to one of those categories
or whether it is an identifier:
\begin{minted}{javascript}
  let type;
  if (booleans.includes(name)) type = 'boolLiteral';
  else if (keywords.includes(name)) type = 'keyword';
  else type = 'identifier';
\end{minted}

Finally, we can push a token with the type we determined for the token:
\begin{minted}{javascript}
  tokens.push({ type, value: name });
  continue;
\end{minted}


\subsection{Line comments}
For detecting line comments, we use the regular loop structure.
The initial detection checks for the semicolon character:
\begin{minted}{javascript}
  if (char === ';') {
    // ...
  }
\end{minted}

The loop runs until the end of the line by checking for line breaks:
\begin{minted}{javascript}
  do {
    // ...
  } while (char !== '\n');
\end{minted}
Line break characters may vary depending by operating system,
but we will assume \code{\n} for the sake of simplicity
and stay focused on the actual formatting algorithms.

After the loop, we push a token as usual and continue with the main loop.


\subsection{Block comments}
The final block of code for lexing detects block comments.
Those are delimited by two characters (\code{#| ... |#}),
so we perform two checks before entering the loop:
\begin{minted}{javascript}
  if (char === '#') {
    let commentChar = code[++position];
    if (commentChar === '|') {
      // ...
    }
  }
\end{minted}
If the inner check fails, we will fall through to the throw statement
at the bottom of the lexer main loop and report that
the character after the \code{#} is invalid.

If both conditions are truthy, we have detected a block comment.
We initialize the variable to store its text
with the character \code{#} instead of an empty string,
because we have already advanced the cursor beyond its occurrence in the code,
so the loop will start at the pipe character (\code{|}):
\begin{minted}{javascript}
  let text = '#';
\end{minted}

The exit condition for the loop is the detection
of the block comment terminator (\code{|#}),
which will be the last bit of code included in the text of the token:
\begin{minted}{javascript}
  do {
    text += commentChar;
    commentChar = code[++position];
  } while (!text.endsWith('|#'));
\end{minted}

The block ends by pushing a new token and continuing the main loop:
\begin{minted}{javascript}
  tokens.push({ type: 'blockComment', value: text });
  continue;
\end{minted}

