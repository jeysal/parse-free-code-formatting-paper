\subsection{Block comments}
The final block of code for lexing detects block comments.
Those are delimited by two characters (\code{#| ... |#}),
so we perform two checks before entering the loop:
\begin{minted}{javascript}
  if (char === '#') {
    let commentChar = code[++position];
    if (commentChar === '|') {
      // ...
    }
  }
\end{minted}
If the inner check fails, we will fall through to the throw statement
at the bottom of the lexer main loop and report that
the character after the \code{#} is invalid.

If both conditions are truthy, we have detected a block comment.
We initialize the variable to store its text
with the character \code{#} instead of an empty string,
because we have already advanced the cursor beyond its occurrence in the code,
so the loop will start at the pipe character (\code{|}):
\begin{minted}{javascript}
  let text = '#';
\end{minted}

The exit condition for the loop is the detection
of the block comment terminator (\code{|#}),
which will be the last bit of code included in the text of the token:
\begin{minted}{javascript}
  do {
    text += commentChar;
    commentChar = code[++position];
  } while (!text.endsWith('|#'));
\end{minted}

The block ends by pushing a new token and continuing the main loop:
\begin{minted}{javascript}
  tokens.push({ type: 'blockComment', value: text });
  continue;
\end{minted}
