\subsection{Comments}
The treatment of comments varies significantly for different lexers.
Lexers used in compilers often discard them right away,
but for the purpose of code formatting, they need to be preserved
and must thus appear in the token stream like any other token.
As such, we can assign formatting rules to them like we did for
identifiers, prefixes, \code{leftPar}s and other token types.

For line comments (\code{;}), we must place a hard line break behind them,
so we do not risk reformatting
\begin{minted}{lisp}
  (a ; comment
  b)
\end{minted}
to
\begin{minted}{lisp}
  (a ; comment b)
\end{minted}
, modifying the program syntax and in this case making it invalid.
Block comments (\code{#| |#}) do not carry this risk.
Spaces around comments, as well as potential soft breaks, are desired.

Using these rules, we can output comments in a variety of ways
depending on the context they appear in, for both line comments:
\begin{minted}{lisp}
  (a b) ; at the end of a line

  (a b cdefghijklmnopqrstuvwxyz)
  ; on their own after a used soft line break
\end{minted}
, and block comments:
\begin{minted}{lisp}
  (a b c
  #| at the beginning of a line |# d e)

  (a b c #| or right in the middle |# d e)

  (a b c
  #| spanning
  multiple
  lines |#
  d e)
\end{minted}

We could go further and convert line breaks inside block comments to hard line breaks
and split the block comment into multiple tokens of type block comment,
while only the first keeps the starting comment delimiter \code{#|},
and only the last keeps the terminating comment delimiter \code{|#}.
This would have the effect that our printer understands that the comment
is not entirely on one line, and it might in the multi-line example above print
\begin{minted}{lisp}
  (a b c #| spanning
  multiple
  lines |# d e)
\end{minted}
instead, using some more of the available space
on the first and last lines of the block comment for the remaining tokens.

It would also cause the printer to apply indentation inside the block comment.
Previously, indentation was primarily relevant after soft line breaks,
because our hard line breaks occurred in locations
where the nesting and indentation were zero.
With hard line breaks from block comments appearing in the middle of expressions,
we could profit from the existing indentation algorithm to align comments appropriately:
\begin{minted}{lisp}
  (a (b (c (d (e (f (g
              h #| multi
              line
              comment |# )))))))
\end{minted}
looks a lot better than:
\begin{minted}{lisp}
  (a (b (c (d (e (f (g
              h #| multi
  line
  comment |# )))))))
\end{minted}

It is possible to perform some more complex formatting on comments, such as
indenting text inside of block comments and
inserting soft line breaks inside of block comments and even
converting line comments to block comments to avoid violating the line length limit,
but we will not deal with these advanced operations in the scope of this paper.
