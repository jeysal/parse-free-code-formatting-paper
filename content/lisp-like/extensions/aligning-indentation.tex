\subsection{Aligning indentation style}

When discussing our method for indenting code in this Lisp-like language,
we noticed that the example code for an implementation of the factorial function
used two different styles of indentation.
\input{content/lisp-like/factorial-example.tex}
The middle line is indented according to the depth of expression nesting
and matches the indentation that our algorithm would have assigned to the line.
The last line, however, is indented with six spaces instead of the four
we would expect if the same indentation style had been used.

This alternative indentation style uses the column where the
second element of a list started as a baseline for any further elements
that are part of the same list and need to be assigned an indentation level.
The following example uses this style consistently,
albeit containing strangely placed line breaks that cause the weird appearance:
\begin{minted}{lisp}
  (a b c
     d (e f
          g) (h i
                j (k l
                     m)
                n)
     o)
\end{minted}
Using spaces for indentation is mandatory with this style,
since tabs cannot provide sufficiently granular control to achieve exact alignment.

To implement this method of indentation, we can use a stack (\textit{LIFO}) data structure
containing numeric values that describe the indentation level per nesting level.
\begin{itemize}
  \item When printing a \code{leftPar}, we push a 0 onto the stack.
  \item When printing a token after a space and the stack holds a 0,
  we replace the 0 with the column where the token starts.
  \item When printing a \code{rightPar}, we pop a number off the stack.
\end{itemize}
Whenever we need to look up the indentation level required for the current position,
we can just look it up on top of the stack.

This is the example from above, annotated at each line break
with the current stack values as zero-based column indices:
\begin{minted}{lisp}
  ;()
  (a b c
  ;(3)
     d (e f
  ;(3, 8)
          g) (h i
  ;(3, 14)
                j (k l
  ;(3, 14, 19)
                     m)
  ;(3, 14)
                n)
  ;(3)
     o)
  ;()
\end{minted}

\paragraph{Early line breaks}
It is possible that we need to provide an indentation level
before encountering the second element of the current list,
when the stack still contains a zero at the top.
This could happen if there is a line break immediately after the
\code{leftPar} and the first identifier:
\begin{minted}{lisp}
  (aaa
  b c) ; how to indent this line?
\end{minted}
It is even more likely to occur if the first expression
is more complex than a single identifier:
\begin{minted}{lisp}
  ((a1 a2 a3
       a4 a5)
  b c) ; how to indent this line?
\end{minted}
There are multiple reasonable options to handle this case, including
\begin{itemize}
  \item using the indentation level from the previous nesting level, plus an offset, and
  \item aligning with the \textit{first} element of the list instead.
\end{itemize}

\paragraph{Escalating indentation}
It might also be a good idea to use this indentation style only if
the indentation level does not jump up more than a certain threshold,
and fall back to the nesting-based style otherwise.
The pure form of this indentation method can
provoke extremely deep indentation very quickly:
\begin{minted}{lisp}
  (firstElementIsVeryLong second
                          ((first element is complex) second
                                                      third))
\end{minted}
It is also dangerous when paired with certain soft line breaking algorithms,
such as the greedy line-filling method we defined:
\begin{minted}{lisp}
  ; line size 20
  (aaaa (bbbb cccc
              dddd))
\end{minted}
