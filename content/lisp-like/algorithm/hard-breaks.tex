\subsection{Hard line breaks}
The only source code position where we always want to break
is after the completion of a top-level expression.
We can use the same counter we introduced for indentation for this purpose as well
by inserting a hard line break after a \code{rightPar} token if and only if
the counter has been decremented to 0.
This way, the tokens
\begin{minted}{text}
  (, define, x, (, +, 1, 2, ), ), (, display, x, )
\end{minted}
will be printed as
\begin{minted}{scheme}
  (define x (+ 1 2))
  (display x)
\end{minted}
, with hard breaks inserted after the second and third \code{rightPar}, but not the first.

We can also use hard line breaks to preserve empty lines
that may have been present in the original source code.
This feature has already been mentioned in the introduction and
is usually indispensable for any practical formatter,
because developers will need those empty lines in their code to structure it clearly.
However, this does require changes to the lexer,
so it can already insert hard line break tokens into the token stream
whereever it finds a line that does not contain any token.
Our counter-based hard break insertion will also need to take into account
that it might find hard breaks already present in the token stream
and must not insert another one if the \code{rightPar} is already followed by one,
otherwise we might end up printing more than one consecutive empty line.
