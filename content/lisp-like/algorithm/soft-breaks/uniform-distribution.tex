\paragraph{Uniform length distribution}
There is still one strategy for placing soft line breaks that we disregarded so far:
Distributing the tokens so each partial line is about the same in size.
There are multiple ways to implement this strategy.
Unfortunately, none of them is computationally quite as trivial
as the greedy line-filling algorithm we discussed before ---
going through the tokens left to right once will not suffice here.
For a solution that is still quite fast and simple,
we can split a line that is too long between its two most central tokens.
For example, the 21-characters long (after spacing) line
\begin{minted}{lisp}
  (a bc def ghij klmno)
\end{minted}
would be split into
\begin{minted}{lisp}
  (a bc def
  ghij klmno)
\end{minted}
, with the partial lines being 9 and 11 characters long (before indentation).
We then need to recalculate the indentation for the part after the split.
Taking the indentation into account, we determine for both parts whether they
still exceed the length limit, and for each one that does, repeat the split recursively.

Just like with greedy line-filling, we can treat tokens without a space between them
as one token to avoid undesired separation of them.
Inevitable length limit violations are not as easy to deal with,
because by the time we detect such an offense, we are at a leaf of the recursion tree.
We could extend the algorithm to give feedback on
how many columns of horizontal space a partial line ended up using,
so further invocations know that they can use that extra space.

Note that this algorithm is not optimal in distributing the length uniformly.
Consider the tokens:
\begin{minted}{text}
  (, aaa, b, c, dd, )
\end{minted}
Given line size 6, our algorithm would insert a soft line break between
\code{b} and \code{c}, and then between \code{c} and \code{dd},
producing the code
\begin{minted}{lisp}
  (aaa b
    c
    dd)
\end{minted}
as opposed to the optimal length distribution:
\begin{minted}{lisp}
  (aaa
    b c
    dd)
\end{minted}

It may prove useful to pick a slightly higher ratio than 50\% to split at.
This is because practical lisp code tends to have higher levels of nesting
towards the end of an expression, so we can expect the code after the split
to have more indentation whitespace that occupies some extra columns.
