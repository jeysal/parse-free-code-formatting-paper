\paragraph{Inevitable length limit violation}
Excessive expression nesting leading to deep indentation
that consumes most of the available columns
or simply the occurence of overly large tokens
can force us to violate the restrictions on maximum line length.
Even if we ignore the separation avoidance mechanism we just introduced,
there is no way, including breaking at every single possible position,
to place soft line breaks so the following code does not exceed the line length limit:
\begin{minted}{lisp}
  ; limit -|
  (
    abc
    (
      def
      (
        ghi
        (
          jkl
        )
      )
    )
  )
\end{minted}
We could limit the width of indentation that we print to make this problem less likely to occur.
But we cannot entirely eliminate the problem this way ---
if we have to print a large token, the line length may still be exceeded,
even without extraordinarily deep indentation:
\begin{minted}{lisp}
  ; limit 25 -------------|
  (
    abc
    (
      someVeryLargeIdentifier
    )
  )
\end{minted}

To mitigate the effect of such a violation and still
print the remainder of the tokens in a reasonable way,
we bump the maximum line length up to match
the overhang we had to use for the offending token
until the next hard line break.
The following example is once again printed with our regular algorithm,
including separation avoidance.
\begin{minted}{lisp}
  ; limit 15 ---|
  (abc 12345
    12345
    largeIdentifier
    0123456 0123456
    0123456
    0123456)
\end{minted}
The second \code{12345} does not fit into the same line anymore and is placed on a new line instead.
\code{largeIdentifier} does not fit into that line, so it is placed on a new line as well.
However, even on that new line,
the token takes up too much horizontal space, overhanging by 2 characters.
We have no choice but to print it like that and to temporarily bump our line length to 17.
Because of the increased line length, the following line can hold two \code{0123456} tokens,
which would otherwise have been printed on separate lines.
The final two \code{0123456}s cannot be printed on one line,
because we cannot separate the latter one from its closing \code{rightPar},
which in turn cannot be printed on the same line because it would occupy column 18,
exceeding even our increased line length.

Retrospectively, we could have also printed both \code{12345}s on the same line,
because our line length for this part of the code ended up being 17 instead of 15,
but extending the algorithm to do this would require giving up the linear processing order
and going back to a token we have already printed.
