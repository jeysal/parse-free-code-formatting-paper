\paragraph{Greedy line-filling}
Another one of the strategies, breaking as late as possible,
is trivial to achieve and requires no more than one pass through the tokens in the line.
In general, however, it does not print very well readable code.
The indentation rules we have already established will take care of inserting
the correct number of spaces or tabs whenever we insert a break.
We keep track of the already printed line length by initializing it with the
number of columns already filled out by the indentation and adding the number
of characters in any token we print to it.
We only print a token if it would not increase our column counter beyond
the maximum allowed line length.
Otherwise, we insert a line break and reset the counter, indenting as usual.
Then we proceed with the remaining tokens until the next hard break.

We also need to carefully consider the interaction with the other printing concerns.
If we break between two tokens that were separated by a space,
the line break must \textit{replace} that space.
We can neither print the space before the line break, producing a line with a trailing space,
nor can we print it after the line break, increasing the indentation of the new line
beyond the indentation generated from our nesting counter.
Concerning indentation, we must additionally take into account that \code{rightPar}s
may be printed as the first token after a soft line break,
and that we defined those to decrease the indentation level \textit{before} being printed.
This means we must process not only the \textit{after}-effect on indentation of the previous token,
but also the \textit{before}-effect of the following token \textit{before}
printing the indentation whitespace that a new line after a soft break requires.

This approach will ensure that too many tokens between two hard breaks will be spread
across multiple lines, and it may do so in a reasonable way,
for example when printing the following tokens with line size 25:
\begin{minted}{text}
  (, let, something, ), (, identifier, 012345,
  43210, (, someIdentifier, 0123456, 789, ), )
\end{minted}
The resulting code with our standard indentation, spacing and hard breaking method
and this greedy line-filling algorithm for soft breaks is:
\begin{minted}{lisp}
  ; line size 25
  (let something)
  (identifier 012345 43210
    (someIdentifier 0123456
      789))
\end{minted}

While this may seem like reasonable formatting style,
it is really just a lucky token arrangement and line length.
If we increase the line length by 1, the code we produce looks weird:
\begin{minted}{lisp}
  ; line size 26
  (let something)
  (identifier 012345 43210 (
      someIdentifier 0123456
      789))
\end{minted}
If we keep the line length but replace \code{identifier}
with \code{something}, something similar happens:
\begin{minted}{lisp}
  ; line size 25
  (let something)
  (something 012345 43210 (
      someIdentifier
      0123456 789))
\end{minted}

\subparagraph{Avoiding undesired separation}
The most annoying issue about both of the `weird' code formats we produced is that
\code{someIdentifier} is separated from its preceding \code{leftPar} by a line break,
especially because the same thing is not applied to the code snippets
\code{(identifier} or \code{(something} that occur earlier and have the same structure.
We want to avoid the separation of token sequences like \code{(x}
and its counterpart \code{y)}, including the case where multiple subsequent
parentheses occur: \code{((x} or \code{y))}.
We also want to avoid a similar separation issue if a break
is inserted immediately after a prefix token,
so constructs like \code{&x} always remain together.
The latter would not only look terrible, it may also be forbidden by some languages
with special treatment of prefixes.

We notice that for the language we are dealing with,
the token pairs that must not be separated by line breaks
are exactly those token pairs that do not allow the insertion of a space between them.
Because of this, we can base a system of line break allowance on our established spacing rules,
effectively treating tokens that are printed with no space inbetween them as single tokens.

This makes the two problematic examples more bearable
and in this case even uniform among one another
as well as consistent with the unproblematic example:
\begin{minted}{lisp}
  ; line size 26
  (let something)
  (identifier 012345 43210
    (someIdentifier 0123456
      789))

  ; line size 25
  (let something)
  (something 012345 43210
    (someIdentifier 0123456
      789))
\end{minted}

\subparagraph{Inevitable length limit violation}
Excessive expression nesting leading to deep indentation
that consumes most of the available columns
or simply the occurence of overly large tokens
can force us to violate the restrictions on maximum line length.
There is no way to place soft line breaks so
the following code does not exceed the line length limit,
including breaking at every single possible position,
even if we ignore the separation avoidance we just introduced:
\begin{minted}{lisp}
  ; limit -|
  (
    abc
    (
      def
      (
        ghi
        (
          jkl
        )
      )
    )
  )
\end{minted}
We could limit the width of indentation that we print to make this problem less likely to occur.
But we cannot entirely eliminate the problem this way ---
if we have to print a large token, the line length may still be exceeded,
even without extraordinarily deep indentation:
\begin{minted}{lisp}
  ; limit 25 -------------|
  (
    abc
    (
      someVeryLargeIdentifier
    )
  )
\end{minted}

To mitigate the effect of such a violation and still
print the remainder of the tokens in a reasonable way,
we bump the maximum line length up to match
the overhang we had to use for the offending token
until the next hard line break.
The following example is once again printed with our regular algorithm,
including separation avoidance.
\begin{minted}{lisp}
  ; limit 15 ---|
  (abc 12345
    12345
    largeIdentifier
    0123456 0123456
    0123456
    0123456)
\end{minted}
The second \code{12345} does not fit into the same line anymore and is placed on a new line instead.
\code{largeIdentifier} does not fit into that line, so it is placed on a new line as well.
However, even on that new line,
the token takes up too much horizontal space, overhanging by 2 characters.
We have no choice but to print it like that and to temporarily bump our line length to 17.
Because of the increased line length, the following line can hold two \code{0123456} tokens,
which would otherwise have been printed on separate lines.
The final two \code{0123456}s cannot be printed on one line,
because we cannot separate the latter one from its closing \code{rightPar},
which in turn cannot be printed on the same line because it would occupy column 18,
exceeding even our increased line length.

Retrospectively, we could have also printed both \code{12345}s on the same line,
because our line length for this part of the code ended up being 17 instead of 15,
but extending the algorithm to do this would require giving up the linear processing order
and going back to a token we have already printed.
