\paragraph{Greedy line-filling}
Another one of the strategies, breaking as late as possible,
is trivial to achieve and requires no more than one pass through the tokens in the line.
In general, however, it does not print very well-readable code.
The indentation rules we have already established will take care of inserting
the correct number of spaces or tabs whenever we insert a break.
We keep track of the already printed line length by initializing it with the
number of columns already filled out by the indentation and adding the number
of characters in any token we print to it.
We only print a token if it would not increase our column counter beyond
the maximum allowed line length.
Otherwise, we insert a line break and reset the counter, indenting as usual.
Then we proceed with the remaining tokens until the next hard break.

We also need to carefully consider the interaction with the other printing concerns.
If we break between two tokens that were separated by a space,
the line break must \textit{replace} that space.
We can neither print the space before the line break, producing a line with a trailing space,
nor can we print it after the line break, increasing the indentation of the new line
beyond the indentation generated from our nesting counter.
Concerning indentation, we must additionally take into account that \code{rightPar}s
may be printed as the first token after a soft line break,
and that we defined those to decrease the indentation level \textit{before} being printed.
This means we must process not only the \textit{after}-effect on indentation of the previous token,
but also the \textit{before}-effect of the following token \textit{before}
printing the indentation whitespace that a new line after a soft break requires.

This approach will ensure that too many tokens between two hard breaks will be spread
across multiple lines, and it may do so in a reasonable way,
for example when printing the following tokens with line size 25:
\begin{minted}{text}
  (, let, something, ), (, identifier, 012345,
  43210, (, someIdentifier, 0123456, 789, ), )
\end{minted}
The resulting code with our standard indentation, spacing and hard breaking method
and this greedy line-filling algorithm for soft breaks is:
\begin{minted}{lisp}
  ; line size 25
  (let something)
  (identifier 012345 43210
    (someIdentifier 0123456
      789))
\end{minted}
