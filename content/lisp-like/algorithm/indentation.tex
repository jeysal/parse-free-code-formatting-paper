\subsection{Indentation}
To achieve an indentation similar to that of given example,
we can keep track of the level of nested parentheses surrounding us using a counter.
That counter is initialized with 0,
incremented by 1 after we pass a \code{leftPar} and
decremented by 1 before we pass a \code{rightPar}.
After a line break is placed, we can determine the size of the whitespace we need to print
by looking at the current state of the counter \textit{n}.
We print \textit{n} tabs if our type of indentation is tabs,
or \textit{n * m} spaces if our type of indentation is spaces,
\textit{m} being a positive integer denoting how large our indentation steps should be.
In the code examples shown here, we indent with two spaces each.

We have not yet introduced a mechanism for inserting line breaks between tokens
so everything would be printed in a single line and there is nothing to indent yet,
but for visualization purposes we will assume that line breaks will be placed
some particular spots and represent them as special hard break tokens.

If we use the counter method to generate indentation for the tokens
\begin{minted}{text}
  (, avg, num1, <HARDBREAK>,
  (, add, num2, <HARDBREAK>,
  num3, ), <HARDBREAK>,
  num4, )
\end{minted}
, we get the printed result:
\begin{minted}{lisp}
  (avg num1
    (add num2
      num3)
    num4)
\end{minted}
This format is well-readable in that the indentation and dedentation make it
quite easy to see which expression each identifier belongs to.
It also saves a lot of space by never indenting more than required by the level of expression nesting,
so we will not have to split up lines too often when dealing with the soft line breaking aspect.

It is, however, not exactly the same formatting style that we see in the `factorial' example.
Using our method, we would get the following code for the factorial function,
again assuming that the line breaks remain the same:
\begin{minted}{lisp}
  (defun factorial (n)
    (if (= n 0) 1
      (* n (factorial (- n 1)))))
\end{minted}
Comparing this to the original factorial example code
\begin{minted}{lisp}
  (defun factorial (n)
    (if (= n 0) 1
        (* n (factorial (- n 1)))))
\end{minted}

, we notice that while the first two lines are indented in the same way that we managed to reproduce,
but the last line is indented by six spaces instead of four.
Here, another indentation method is applied,
one that aligns the start of the second and all following elements of a list
by setting the indentation of the third and all following elements
to the column where the second element was printed.
This results in deeper indenting that takes up more space,
but might be considered more readable by some.
Unlike the example code, we will have to choose one of the two styles of indenting,
but we can alternatively produce the latter `aligning' style as well and will do so later in the chapter.

Finally, let us briefly clarify why we need to decrement the counter \textit{before} a \code{rightPar},
although we increment it \textit{after} a \code{leftPar}.
This is only relevant when there is a line break right before a \code{rightPar}.
Consider the following code:
\begin{minted}{lisp}
  (add someLargeIdentifier someVeryLargeIdentifier)
\end{minted}
If our line length is limited to 25 characters,
the only way to print the code without violating the limit is
\begin{minted}{lisp}
  ; line size 25
  (add someLargeIdentifier
    someVeryLargeIdentifier
  )
\end{minted}
, putting the \code{rightPar} on a new line
because it would be one character too much
to appear immediately after \code{someVeryLargeIdentifier}.
If this happens, we want the \code{rightPar} to be aligned with the corresponding \code{leftPar}.
Not doing this would look particularly weird if there are deeper nested expressions,
since the outer \code{rightPar} would appear as if it belonged to the inner \code{leftPar}:
\begin{minted}{lisp}
  ; line size 30
  (add someLargeIdentifier
    (sub someNestedIdentifier
      someOtherNestedIdentifier)
    )
\end{minted}
