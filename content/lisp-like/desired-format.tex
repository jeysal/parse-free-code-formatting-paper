\section{Desired code format}
These tokens are for example sufficient to build the factorial algorithm
from the Lisp programming language Wikipedia page:
\begin{minted}{lisp}
  (defun factorial (n)
    (if (= n 0) 1
        (* n (factorial (- n 1)))))
\end{minted}


We would consider this code example well-formatted ---
the spacing is consistent,
nested expressions are indented and
line breaks are used to split up the top-level expression
that would otherwise be too long to comprehend it quickly.

One might also notice that there are no inherent differences in formatting
between identifiers, keywords, operators and literals.
This is somewhat consistent with their treatment in code,
where they are all conceptually just elements of a list that is denoted by an
S-expression and they can often be used interchangably.
This realization of course further simplifies the formatting of an already simple language.
