\chapter{Conclusion}
We have learned about general-purpose and formatter-specific
lexing, parsing and printing as operations that convert between
source code, tokens and syntax trees in Chapter \ref{chap:ordinary}.
For the essential printing operation that distinguishes formatting algorithms,
we extracted four key aspects:
\begin{enumerate}
  \item{\textit{spacing}}
  \item{\textit{indentation}}
  \item{\textit{hard line breaks}} and
  \item{\textit{soft line breaks}}.
\end{enumerate}
The separation of these concerns proved to be a very useful tool in
handling the complexity of printing code and
allowing for an incremental approach to
the conception and implementation of an algorithm.

In Chapter \ref{chap:eliminate-parsing},
we uncovered the reasons why
pretty formatting usually requires parsing and
omitting this very step implies incurring severe drawbacks.
Weighing those drawbacks up against the benefits,
we determined a few select use cases where
parse-free formatting may actually be a good idea.

We developed an algorithm to print the
tokens of a Lisp-like language in Chapter \ref{chap:lisp-like},
taking the four aspects of printing into account.
Spacing rules were easily dealt with,
although this may be a property of Lisp-like languages
that lack advanced concepts such as
unary versus binary operations.
Hard line breaks were not difficult to handle either,
but that too may change with concepts
such as block statements and control flow statements.
Indentation left some room for interpretation
with no apparent right way of doing it.
Unsurprisingly, soft line breaks turned out to be the hardest aspect,
both in the variety of possible approaches
and the quality level achievable with limited syntactic information.

% we took that and made an implementation for it. the example from that shows...
We took our algorithmic groundwork and
made a formatter implementation out of it in Chapter \ref{chap:implementation}.
We managed to realize the complete process of formatting
with moderately complex and lengthy code,
finally enabling us to probe parse-free formatting in practice.
The given example shows that previously unformatted code
is printed out in a significantly more readable,
albeit not perfect, way.
The spacing and indentation is appropriate,
but soft line breaks could be placed much better;
they often cut right through expressions with our implementation.

Code that was already formatted by a human, however,
becomes significantly less readable
because the soft line breaks are placed
without awareness of the structure of the expressions
formed by the surrounding tokens.
This is not to blame on omitting the separation avoidance algorithm ---
in the example code, the relevant tokens were rarely separated anyway.
Trying to break at a high point in the syntax tree
like many established algorithms do
may have helped with this issue.

We have managed to implement the basics of formatting without a parser.
The example code had its empty lines preserved;
comments and hard breaks are in the right places.
The indentation works fine, but it is ruined by
the soft line breaks that we did not manage to place sensibly
for our Lisp-like language.
If we want do better in this aspect,
we will probably need to look at the expression tree structure
in a way that essentially equates to parsing
or at least give up on the strict approach
of never looking at more than two consecutive tokens
and never backtracking in the token stream.

It would be interesting to implement a
less minimalistic parse-free formatter
for a real, widely used programming language.
Such a formatter could be compared with
existing formatters for the same language
too evaluate how significant the performance benefit is.
Some language could perhaps be
better formatted than a Lisp dialect with this method,
because they have many small statements
instead of the few large expressions from Lisp,
reducing the impact of a suboptimal soft line breaking algorithm.
It might also make sense to implement a formatter
that specifically targets the use cases
we expect a parse-free routine to be most suitable for
--- large batches of minified code.
