\chapter{Conclusion}
The given example shows that previously unformatted code
is printed out in a significantly more readable,
albeit not perfect, way.
The spacing and indentation is appropriate,
but soft line breaks could be placed much better;
they often cut right through expressions with our implementation.

Code that was already formatted by a human, however,
becomes significantly less readable
because the soft line breaks are placed
without awareness of the structure of the expressions
formed by the surrounding tokens.
This is not to blame on omitting the separation avoidance algorithm ---
in the example code, the relevant tokens were rarely separated anyway.
Trying to break at a high point in the syntax tree
like many established algorithms do
may have helped with this issue.

We have managed to implement the basics of formatting without a parser.
The example code had its empty lines preserved;
comments and hard breaks are in the right places.
The indentation works fine, but it is ruined by
the soft line breaks that we did not manage to place sensibly
for our Lisp-like language.
If we want do better in this aspect,
we will probably need to look at the expression tree structure
in a way that essentially equates to parsing
or at least give up on the strict approach
of never looking at more than two consecutive tokens
and never backtracking in the token stream.

It would be interesting to implement a
less minimalistic parse-free formatter
for a real, widely used programming language.
Such a formatter could be compared with
existing formatters for the same language
too evaluate how significant the performance benefit is.
Some language could perhaps be
better formatted than a Lisp dialect with this method,
because they have many small statements
instead of the few large expressions from Lisp,
reducing the impact of a suboptimal soft line breaking algorithm.
It might also make sense to implement a formatter
that specifically targets the use cases
we expect a parse-free routine to be most suitable for
--- large batches of minified code.
