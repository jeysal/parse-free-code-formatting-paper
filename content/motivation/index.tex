\chapter{Motivation}
\textit{Code formatting}, also known as \textit{code beautification} or \textit{prettyprinting},
has long been a staple in software development tooling.
Researchers have noticed its necessity and started working on solutions decades ago,
\autocite{syntaxDirectedPrettyprinting}
and the widespread availability of such coding assistance has increased ever since.

Some languages --- such as Java --- are commonly written in an integrated development environment (IDE).
The emergence of those full-fledged tooling packages has helped render formatting tools ubiquitous.
Other languages --- such as JavaScript --- are more commonly written using a rather lightweight editor,
often because the language is less verbose or simply for historical reasons.
But even some of those editors --- such as the web development-focused Microsoft Visual Studio Code ---
now include formatting tools in their core distributions together with
Git version control integration, syntax highlighting and other tooling that is deemed essential,
disposing the need to install an additional plugin that provides code formatting support.

In the following chapter, we will roughly outline how code formatters typically work.
Once we have established fundamental comprehension of their technical background,
we can then move on to propose an alternative method that relinquishes parsing.
We address potential benefits and uses, but also drawbacks of this method,
and afterwards develop its implementation for a concrete language.
