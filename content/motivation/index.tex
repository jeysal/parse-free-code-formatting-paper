\chapter{Motivation}
\textit{Code formatting}, also known as \textit{code beautification} or \textit{prettyprinting},
has long been a staple in software development tooling.
Researchers have noticed its necessity and started working on solutions decades ago,
\autocite{syntaxDirectedPrettyprinting}
and the widespread availability of such coding assistance has increased ever since.

Some languages --- such as Java --- are commonly written in an integrated development environment (IDE).
The emergence of those full-fledged tooling packages has helped render formatting tools ubiquitous.
Other languages --- such as JavaScript --- are more commonly written using a rather lightweight editor,
often because the language is less verbose or simply for historical reasons.
But even some of those editors --- such as the web development-focused Microsoft Visual Studio Code ---
now include formatting tools in their core distributions together with
Git version control integration, syntax highlighting and other tooling that is deemed essential,
disposing the need to install an additional plugin that provides code formatting support.
\autocite{vsCodeJavascriptDoc}

These formatters usually work by lexing and parsing the input code
in a process similar to the one a compiler for the same language would utilize,
but then instead of proceeding to transform or semantically analyze the generated syntax tree,
they use the syntactic information obtained by parsing
to print out the token stream obtained by lexing
in a defined format that humans consider readable and comprehensible,
optionally preserving the shape of the original code to some degree.
\autocite{prettyGoodFormattingPipeline}

While many of the algorithms used in common code formatters
operate on plain streams of tokens, \autocite{prettyprinting}
they also require contextual information
to recognize the meaning of tokens and groups of tokens
and determine appropriate positions for the insertion of spaces or line breaks.
Parsing provides that information, which we can build smart printing routines on,
but it is also an expensive operation on top of the aforegoing lexing
and forces us to couple our formatter implementation more tightly
to the syntactical grammar and thus the concrete target language.

We introduce a procedure that skips the parsing step and
build a formatter that can move on from the lexing step straight to printing.
This procedure will generate less information about the code we are operating on and
thus necessarily limit our capability to generate code
formatted in a consistent and customizable output format.

At the same time, our formatter will become quite easy to adapt to vastly different languages,
because we only need to know its token types instead of the entire syntactic grammar and
we can potentially achieve higher formatting performance and start emitting output earlier
because the necessity for an initial parsing step no longer applies.

In Chapter \ref{chap:ordinary}, we will roughly outline how code formatters typically work.
Once we have established fundamental comprehension of their technical background,
we can then move on to propose an alternative method that relinquishes parsing.
We address potential benefits and uses,
but also drawbacks of this method in Chapter \ref{chap:eliminate-parsing}.

Based on a simple Lisp-like target language,
we then design a parse-free printing algorithm
that deals with all the major aspects of code formatting
and gracefully handles most edge cases
in Chapter \ref{chap:lisp-like}.
The very same algorithm then serves as the theoretical foundation
that we build a complete implementation of a formatter
for the language upon and test it with example code input
in Chapter \ref{chap:implementation}.

The goal is for this implementation to tokenize code in the given Lisp-like language
and print it back out with certain guarantees on
spacing based on surrounding token types,
indentation based on expression nesting and
line breaks based on expression nesting, line lengths and the original input code shape.
