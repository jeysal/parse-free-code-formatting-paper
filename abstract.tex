\begin{abstract}
  \addcontentsline{toc}{chapter}{Abstract}

  {\noindent\large\textbf{Abstract}\par}\noindent
  We propose an approach to code formatting
  based entirely on lexical analysis of the input code,
  disposing the need for expensive syntactic analysis.
  This approach has the potential to
  improve the resource efficiency of a formatter
  and make it easily adaptable to different languages,
  while sacrificing consistency and beauty
  of the output format to some degree.
  We develop a parse-free algorithm that handles
  all essential aspects of code formatting
  for a simple Lisp-like language.
  We implement a lexer and a printer
  adhering to this algorithm.

  \vspace{3cm}
  {\noindent\large\textbf{German Abstract}\par}\noindent
  Wir stellen einen Ansatz zur Codeformatierung vor,
  der vollständig auf lexikalischer Analyse des Inputcodes basiert
  und ohne aufwändige syntaktische Analyse auskommt.
  Dieser Ansatz kann potenziell
  die Ressourceneffizienz eines Formatters verbessern
  und ihn leicht an andere Sprachen anpassbar machen,
  tauscht dafür jedoch im Gegenzug ein gewisses Maß an
  Konsistenz und Ansehnlichkeit des Ausgabeformats ein.
  Wir entwickeln einen Algorithmus
  für eine einfach Lisp-ähnliche Sprache,
  der ohne Parsing alle essentiellen
  Aspekte der Codeformatierung umsetzt.
  Wir implementieren diesen Algorithmus
  in einem Lexer und einem Printer.
\end{abstract}
